\clearpage
\section{Rozwój systemu budowania projektu (AK)}
\label{ch:building_project}

W niniejszej części pracy zawarty został opis najważniejszych modyfikacji wprowadzonych przez autorów w systemie budowania aplikacji i bibliotek projektu GGSS. Przeprowadzone przez autorów prace miały na celu przede wszystkim dalszy rozwój i utrzymanie rozwiązania przygotowanego przez nich w ramach studiów inżynierskich.


\subsection{Wprowadzenie do problematyki}
Obecnie stosowany w projekcie system budowania został w znaczącej części przygotowany przez autorów w ramach napisanej przez nich pracy inżynierskiej. Rozwiązanie to oparte zostało o narzędzie CMake oraz, w mniejszym stopniu, o język programowania Python, i umożliwia poprawne zbudowanie każdego elementu (tzn. aplikacji lub biblioteki) wchodzącego w skład projektu GGSS. Przygotowany w ramach pracy inżynierskiej system miał spełniać następujące założenia:
\begin{itemize}
    \item niezależność od systemu operacyjnego - system powinien działać poprawnie zarówno na urządzeniach wykorzystujących system operacyjny Linux, jak i na komputerach z Windowsem. Z tego też powodu do realizacji zadania wybrane zostało narzędzie CMake.
    \item możliwość budowania projektu o skomplikowanej, hierarchicznej strukturze - takim projektem jest, w swojej obecnej postaci, system GGSS
    \item możliwość budowania każdego z komponentów systemu (tzn. biblioteki lub aplikacji) z osobna - zbudowanie niewielkiej aplikacji pomocniczej nie powinno wymagać budowy całego projektu
    \item budowanie każdego elementu projektu raz - jeśli dana zależność wykorzystywana jest przez kilka komponentów systemu, powinna zostać zbudowana tylko jeden raz - takie rozwiązanie pozwala uzyskać czytelną, jednopoziomową wynikową strukturę katalogów, która ułatwia szybkie znalezienie odpowiedniego pliku wykonywalnego lub biblioteki.
\end{itemize}
Wszystkie z wymienionych założeń spełnione zostały już na etapie tworzenia przez autorów pracy inżynierskiej. W jej ramach otrzymane zostało w pełni działające rozwiązanie, które wykorzystywane jest w swoim środowisku docelowym od niemal dwóch lat. Z tego też powodu w niniejszej pracy zamieszczony został jedynie pobieżny opis systemu - jego szczegółowa wersja zawarta została  w przygotowanej przez autorów pracy inżynierskiej, a jej znajomość nie jest konieczna, by zrozumieć zmiany wprowadzone przez autorów w ramach niniejszej pracy.


% jakie sa pliki 
% jak budowana jest hierarchia

W ramach niniejszej pracy autorzy wprowadzili do stworzonego systemu pomniejsze modyfikacje, mające na celu:
\begin{itemize}
    \item poprawę błędów, które wykryte zostały podczas użytkowania systemu 
    \item dodanie wsparcia dla budowania testów jednostkowych i dokumentacji projektu
    \item ułatwienie pracy z systemem - uczynienie jego interfejsu czytelniejszym poprzez zastosowanie mechanizmu funkcji i makr udostępnianego przez narzędzie CMake
    \item rozbudowa skryptu \lstinline{build.py} znajdującego się w repozytorium \emph{ggss-all}, którego zadaniem jest udostępnienie użytkownikowi prostego interfejsu do budowania wybranych komponentów systemu
\end{itemize}


% W ramach niniejszej pracy autorzy wprowadzili kilka modyfikacji, których zadaniem było utrzymanie i rozbudowa już istniejącego systemu. 



% Głównym założeniem przygotowywanego systemu było umożliwienie budowania projektu o strukturze hierarchicznej, co zostało w pełni zrealizowane w ramach pracy inżynierskiej. W ramach niniejszej pracy magisterskiej system ten uległ jedynie niewielkim modyfikacjom, których zadaniem była poprawa błędów, ułatwienie pracy z systemem oraz wprowadzenie kilku pomniejszych funkcjonalności. Z tego też powodu


% Ponieważ rozumienie szczegółów działania przygotowanego systemu nie jest konieczne, by zrozumieć wprowadzone w ramach niniejszej pracy modyfikacje, autorzy zdecydowali o niezamieszczaniu tutaj jego opisu - 


%\subsection{Motywacja do wprowadzenia zmian}

\subsection{Zastosowanie funkcji i makr narzędzia CMake}

\subsection{Wsparcie dla testów i dokumentacji}

\subsection{Rozbudowa skryptu \emph{build.py}}

