\clearpage
\section{Automatyzacja pracy z submodułami (JC)}
\label{sec:gitio}

Niniejsza część manuskryptu została poświęcona obsłudze, obecnej w projekcie GGSS, wielopoziomowej struktury  opartej o mechanizm \emph{git submodules}. Przedstawione zostały zalety i wady zastosowanego w trakcie pracy inżynierskiej rozwiązania. Omówione zostały zmiany, wprowadzone przez autorów w ramach niniejszej pracy magisterskiej, mające na celu ułatwienie pracy z submodułami. Dodatkowo krótko opisane zostały przygotowane \emph{how-to} oraz praktyki które powinno się stosować pracując z taką architekturą.

\subsection{Wprowadzenie do problematyki}
W trakcie przygotowywania pracy inżynierskiej, a konkretnie wykonywania migracji całego projektu GGSS do systemu kontroli wersji Git, zdecydowano się na wykorzystanie technologii \emph{git submodules}. Ze względu na nacisk na zwiększenie modularyzacji projektu technologia ta idealnie wpasowywała się w docelową architekturę. Zasada działania submodułów jest bardzo zbliżona do dowiązań symbolicznych stosowanych między innymi w systemach UNIX. Zamiast wskazywać na ścieżkę do folderu na lokalnym systemie submoduł wskazuje na ścieżkę do konkretnej wersji repozytorium na zewnętrznym serwerze od którego zależy moduł nadrzędny. Rysunek \ref{fig:submodules_links} przedstawia zasadę działania submodułów oraz wpływ wersjonowania na tenże mechanizm. Wykorzystanie submodułów pozwala na w pełni odseparowaną pracę nad wybranym komponentem systemu. Nie zachodzi konieczność pobierania żadnych dodatkowych plików, czy też zależności w celu zmienienia kodu źródłowego. Rozwiązanie to pozwala też na skorzystanie z bardzo szybkiej inicjalizacji całego projektu jedną komendą, co zostało przedstawione w listingu \ref{lst:initialize}.

\begin{figure}[H]
    \centering
    \includegraphics[width=0.9\textwidth]{submodule_links.pdf}
    \caption{Zasada działania submodułów: dowiązanie w ramach danej wersji repozytorium nadrzędnego wskazuje na konkretną wersję repozytorium podrzędnego.}
    \label{fig:submodules_links}
\end{figure}

\lstinputlisting[
    language=Cmd, 
    caption={Inicjalizacja pełnej struktury projektu jedną komendą.}, 
    label={lst:initialize}
]{4_infrastructure/code_samples/submodules_init.txt}

\subsection{Motywacja do wprowadzenia zmian}
Pomimo wielu aspektów \emph{git submodules}, które bardzo dobrze wpasowały się w kreowaną przez autorów w trakcie pracy inżynierskiej strukturę, z technologią tą związanych jest szereg niedogodności. Pierwszy znaczącym problemem napotkanym w trakcie pracy z submodułami było nietypowe zachowanie repozytoriów w trakcie ich inicjalizacji, a konkretnie automatyczne odłączanie ich od głównej gałęzi (wynika to z faktu, iż śledzenie zależności polega w tym przypadku na zapamiętywaniu identyfikatora konkretnej rewizji, nie zaś informacji o gałęzi). Co więcej praca z submodułami wymaga od programisty zwiększonej czujności oraz stosowania dodatkowych zasad, ponieważ więcej jest miejsc na pomyłkę, co może doprowadzić do niepoprawnego działania wykorzystanych narzędzi. Kolejnym problemem napotkanym w trakcie pracy z submodułami jest czasochłonność niektórych operacji, w szczególności aktualizacji repozytorium na samym spodzie drzewa zależności. Zmiana taka wymaga manualnej aktualizacji po kolei każdego z repozytorium, aż do samej góry tejże struktury, co przedstawia rysunek \ref {fig:submodules_update}. Każda z aktualizacji przedstawiona na wspomnianym rysunku, to tak na prawdę cztery lub więcej koniecznych do wykonania akcji, do których wliczają się: aktualizacja repozytorium podrzędnego, dodanie wszystkich zmian do rejestru odpowiedzialnego za ich śledzenie, utworzenie nowej wersji repozytorium, opublikowanie nowej wersji na zewnętrznym serwerze.

\begin{figure}[H]
    \centering
    \includegraphics[width=\textwidth]{submodules_update.pdf}
    \caption{Przykładowa architektura oparta o submoduły z krokami jakie należy podjąć, aby wprowadzić zmiany na ``najniższym`` poziomie.}
    \label{fig:submodules_update}
\end{figure}


\subsection{Automatyzacja z użyciem dedykowanego narzędzia}
\label{subsec:gitio}
Konieczność wykonywania szeregu powtarzalnych czynności w celu wprowadzenia oraz propagacji zmian w poszczególnych modułach projektu GGSS stanowiła problem, który potencjalnie mógłby pochłonąć bardzo znaczącą ilość czasu, możliwego do przeznaczenia na rozwój samego systemu. Dlatego też już w początkowych tygodniach opisywanych prac autorzy zdecydowali się na przygotowanie, z wykorzystaniem języka Python, aplikacji \emph{gitio}, której zadaniem było rozwiązanie przedstawionego problemu. Ze względu na to, że metadane technologii Git są bardzo złożone, a opanowanie zasad wewnętrznego działania tejże technologii wymagałoby bardzo dużo czasu, skorzystano z dedykowanej biblioteki \cite{gitpython} dostępnej z poziomu języka Python. Argumenty wejściowe przyjmowane przez \emph{gitio} to:
\begin{itemize}
    \item \lstinline{-h, --help} - pozwala na wyświetlanie informacji o przeznaczeniu programu oraz przyjmowanych argumentach wraz z krótkim opisem
    \item \lstinline{-p PATH, --path PATH} - ścieżka do głównego folderu zawierające drzewo repozytoriów do wyrównania
    \item \lstinline{-b BIN, --bin BIN} - ścieżka do aplikacji Git (argument ten jest wymagany jedynie, jeżeli \emph{gitio} nie jest w stanie automatycznie wykryć Git'a)
\end{itemize}

Przed uruchomieniem aplikacji \emph{gitio} należy uprzednio przygotować repozytoria, które mają zostać poddane procesowy wyrównania. W tym celu należy wykonać następujące kroki:
\begin{itemize}
    \item sklonować główne repozytorium - \lstinline{git clone <url>}. W celu poprawnego działania należy sklonować repozytorium z użyciem klucza ssh.
    \item zainicjalizować i zaktualizować wszystkie submoduły za pomocą komendy: \\
        \lstinline{git submodule update --init --recursive}
    \item ustawić główną gałąź na każdym z submodułów za pomocą komendy: \\
        \lstinline{git submodule foreach --recursive "git checkout master"}
\end{itemize}
Zasada działania aplikacji jest stosunkowo prosta, natomiast znacząco ułatwia ona działania z wielopoziomową strukturą opartą o \emph{git submodules}. W pierwszej kolejności \emph{gitio} analizuje strukturę katalogów oraz metadane zawarte w folderach \lstinline{.git}, dzięki czemu jest w stanie zapisać w pamięci zależności między repozytoriami. Następnie przechodząc od samego dołu drzewa zależności, czyli repozytoriów, które nie mają żadnych submodułów, wykonywane są następujące akcje:
\begin{itemize}
    \item aktualizacja rewizji, na które wskazują submoduły do najnowszych dostępnych w zdalnym repozytorium
    \item utworzenie nowej rewizji z zaktualizowanymi submodułami
    \item przekazanie nowej rewizji do podłączonego zewnętrznego repozytorium.
\end{itemize}
W celu zapewnienia bezpieczeństwa \emph{gitio} pozwala na wyrównanie wersji repozytoriów tylko w przypadku, gdy w danym repozytorium nie ma żadnych zmian od wykonania ostatniej rewizji. 

Rysunek \ref{fig:gitlab_gitio} przedstawia przykładową rewizję na portalu GitLab utworzoną z użyciem \emph{gitio}. Wiadomość w ramach utworzonej rewizji, czyli \emph{Automatic repository update}, wskazuje na wykorzystanie \emph{gitio}. Jedyne zmiany jakie jest w stanie wprowadzić \emph{gitio}, to aktualizacja rewizji pod-projektu, co również widoczne jest na załączonym rysunku.

\begin{figure}[H]
    \centering
    \includegraphics[width=\textwidth]{gitio_gitlab.png}
    \caption{Przykładowa rewizja utworzona z użyciem \emph{gitio}.}
    \label{fig:gitlab_gitio}
\end{figure}


\subsection{Dokumentacja sposobu pracy z submodułami}
Ze względu na to, że praca z repozytoriami posiadającymi submoduły wymaga poświęcenia dodatkowej uwagi oraz stosowania specyficznych praktyk, autorzy postanowili przygotować stosowną dokumentację, w formie plików \emph{README} umieszczonych w repozytorium \emph{ggss-aux}. Jej stosowanie pozwala na uniknięcie poważnych błędów, które mogłyby doprowadzić do niepoprawnego działania systemu kontroli wersji.

Dokumentacja została podzielona na trzy części. Pierwsza z nich odnosi się do poprawnej inicjalizacji repozytoriów z wielopoziomowymi submodułami. Poniżej zaprezentowane zostało krótkie streszczenie zawartych tam informacji:
\begin{itemize}
    \item W pierwszej kolejności należy sklonować odpowiednie repozytorium z zewnętrznego serwera. W celu ułatwienia pracy z submodułami należy akcję tę wykonać za pomocą protokołu SSH wraz z przypisanym kluczem. Pozwala to uniknąć wielokrotnego wpisywania loginu oraz hasła przy każdorazowym klonowaniu submodułów. Przykładowa komenda dla repozytorium \lstinline{ggss-all} wygląda następująco: \lstinline{git clone ssh://git@gitlab.cern.ch:7999/atlas-trt-dcs-ggss/ggss-all.git}.
    \item W ramach poprzedniego kroku sklonowane zostało jedynie główne repozytorium, brak jest zawartości katalogów z submodułami. Ich inicjalizacja powinna zostać wykonana komendami \lstinline{git submodule init} oraz \lstinline{git submodule update}, natomiast w celu przyspieszenia tego procesu można wykonać jedną komendę, a mianowicie \lstinline{git submodule update --init --recursive}. Właśnie taka komenda polecana przez przez autorów do pracy z projektem GGSS.
    \item Ze względu na sposób działania submodułów, przed dokonaniem zmian należy jeszcze zmienić gałęzie we wszystkich submodułach na gałąź docelową. Domyślnie inicjalizacja submodułów powoduje, że znajdują się one w stanie \lstinline{detached HEAD}, co oznacza odwołanie do konkretnej rewizji, a nie gałęzi. W celu dokonania zbiorczego ustawienia gałęzi należy wykonać komendę \lstinline{git submodule foreach --recursive "git checkout master"}
\end{itemize}

Kolejna część dokumentacji odnosi się do porad i dobrych praktyk, które należy stosować pracując z infrastrukturą repozytoriów opartą o submoduły. Ostatnią częścią dokumentacji jest natomiast krótki opis, w jaki sposób propagować zmiany w całym projekcie GGSS. Propagacja odbywa się za pomocą wyrównania wersji wskazywanych przez submoduły, a zatem przez odpowiednie skorzystanie z \emph{gitio}. Więcej szczegółów jak skorzystać z \emph{gitio} oraz warunków jakie należy spełnić zostały opisane w sekcji \ref{subsec:gitio}. Treść przygotowanej dokumentacji załączona została do niniejszej pracy w formie dodatku \ref{sec:working-with-git-sudmobules}.
