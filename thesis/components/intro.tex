\chapter{Wstęp}
\label{cha:wstep}

\section{Wprowadzenie do systemu GGSS}

Europejska Organizacja Badań Jądrowych CERN jest jednym z najważniejszych ośrodków naukowo-badawczych na świecie i miejscem rozwoju zarówno fizyki, jak i informatyki. Będąc miejscem powstania technologii takich jak protokół HTTP - Hypertext Transfer Protocol, CERN jest kojarzony dziś przede wszystkim z Wielkim Zderzaczem Hadronów (LHC - Large Hadron Collider) - największym akceleratorem cząstek na świecie. Jednym z pracujących przy LHC eksperymentów jest detektor ATLAS (A Toroidal LHC ApparatuS), pełniący kluczową rolę w rozwoju współczesnej fizyki - przyczynił się on do potwierdzenia istnienia tzw. bozonu Higgsa w 2012 roku. 

Detektor ATLAS zbudowany jest z kilku pod-detektorów, tworzących strukturę warstową. Najbardziej wewnętrzną część stanowi tzw. Detektor Wewnętrzny (Inner Detector), składający się z kolei z trzech kolejnych podsystemów. Jednym z tychże podsystemów, szczególnie istotnym w kontekście niniejszej pracy, jest detektor promieniowania przejścia (TRT - Transition Radiation Tracker). 

System Stabilizacji Wzmocnienia Gazowego (GGSS - Gas Gain Stabilization System) jest jednym z podsystemów detektora TRT, mającym zapewnić jego poprawne działanie. Projekt ten stanowi części systemu kontroli detektora ATLAS (DCS - Detector Control System). W skład systemu GGSS wchodzi zarówno warstwa oprogramowania, jak i szereg urządzeń. Ze względu na jego rolę, jednym z najważniejszych wymagań stawianych przed projektem jest wysoka niezawodność.

W niniejszej pracy autorzy przybliżą najważniejsze zmiany dokonane przez nich w czasie półtorarocznych prac nad rozwojem i usprawnieniem systemu GGSS. Prace obejmują przede wszystkim zmiany w warstwie oprogramowania, mające na celu zarówno wprowadzenia nowych funkcjonalności do systemu, jak również uczynienie go bardziej przystępnym dla korzystających z niego osób, m.in. poprzez automatyzację procesów związanych z cyklem życia oprogramowania (np. tworzenie nowych wydań).


\section{Cel pracy}