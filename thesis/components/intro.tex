\chapter{Wstęp}
\label{cha:wstep}

\section{Wprowadzenie do systemu GGSS}

Europejska Organizacja Badań Jądrowych CERN jest jednym z najważniejszych ośrodków naukowo-badawczych na świecie i miejscem rozwoju zarówno fizyki, jak i informatyki. Będąc miejscem powstania wielu znaczących technologii (m.in. protokół \emph{HTTP} - \emph{Hypertext Transfer Protocol}), CERN kojarzony jest dziś przede wszystkim z Wielkim Zderzaczem Hadronów (\emph{LHC} - \emph{Large Hadron Collider}) - największym akceleratorem cząstek na świecie. Jednym z pracujących przy LHC eksperymentów jest detektor ATLAS (\emph{A Toroidal LHC ApparatuS}), pełniący kluczową rolę w rozwoju współczesnej fizyki - przyczynił się on do potwierdzenia istnienia tzw. bozonu Higgsa w 2012 roku.

Detektor ATLAS zbudowany jest z kilku pod-detektorów, tworzących strukturę warstową. Najbardziej wewnętrzną część stanowi tzw. Detektor Wewnętrzny (ang. \emph{Inner Detector}), składający się z kolei z trzech kolejnych podsystemów. Jednym z tychże podsystemów, szczególnie istotnym w kontekście niniejszej pracy, jest detektor promieniowania przejścia (\emph{TRT} - \emph{Transition Radiation Tracker}).

System Stabilizacji Wzmocnienia Gazowego (\emph{GGSS} - \emph{Gas Gain Stabilization System}) jest jednym z podsystemów detektora TRT, mającym zapewnić jego poprawne działanie. Projekt ten zintegrowany jest z systemem kontroli detektora ATLAS (\emph{DCS} - \emph{Detector Control System}). W skład systemu GGSS wchodzi zarówno warstwa oprogramowania, jak i szereg urządzeń. Ze względu na jego rolę, jednym z najważniejszych wymagań stawianych przed projektem jest wysoka niezawodność.

W niniejszej pracy autorzy przybliżą najważniejsze zmiany dokonane przez nich w czasie półtorarocznych prac nad rozwojem i usprawnieniem systemu GGSS. Prace obejmują przede wszystkim zmiany w warstwie oprogramowania, mające na celu zarówno wprowadzenia nowych funkcjonalności do systemu, jak również uczynienie go bardziej przystępnym dla korzystających z niego osób, m.in. poprzez automatyzację procesów związanych z cyklem życia oprogramowania (np. tworzenie nowych wydań).


\section{Cel pracy}


Niniejsza praca jest kontynuacją rozwoju Systemu Stabilizacji Wzmocnienia Gazowego rozpoczętego w ramach pracy
inżynierskiej o tytule \emph{Rozbudowa i uaktualnienie oprogramowania systemu GGSS detektora ATLAS TRT}. Praca inżynierska skupiała się na aspektach infrastruktury oraz architektury projektu. Przeprowadzono migrację projektu na system kontroli wersji Git. Dokonano przebudowę architektury projektu na bardziej modularną oraz prostszą do zrozumienia. Wprowadzono wiele zmian w projekcie, których celem było udoskonalenie procesu wytwarzania oraz wdrażania oprogramowania w środowisko produkcyjne, np.: wykorzystanie technologii CMake oraz GitLab CI/CD.


Główny nacisk pracy magisterskiej został położony na część aplikacyjną projektu - kod źródłowy odpowiedzialny za
główną logikę został rozbudowany oraz udoskonalony. W ramach zmian w kodzie zostały dodane nowe funkcjonalności,
nieużywany kod został usunięty z projektu, jakość kodu została poprawiona, a jego poprawne działanie zostało
zabezpieczone poprzez testy automatyczne. Ze względu na ciągłą pracę z systemem, poznawanie newralgicznych punktów oraz środowiska, w ramach którego system jest uruchamiany, część pracy magisterskiej zostanie poświęcona kontynuacji prac nad infrastrukturą oraz architekturą. W ramach projektu skupiono się również na aspektach organizacji pracy oraz technik zastosowanych w celu jej poprawienia. Ważnym aspektem pracy, który zostanie uwzględniony w manuskrypcie były zarówno testy automatyczne, testy manualne jak i przygotowanie infrastruktury potrzebnej do ich przeprowadzenia.


Ze względu na bardzo szeroki zakres tematów podejmowanych w tejże pracy zdecydowano się na podział, który odchodzi od standardowego. W celu ułatwienia korzystania z manuskryptu wprowadzenie do opisywanego problemu oraz wykonane prace zostaną zamieszczone w jednym miejscu. Zatem zarówno nakreślenie problemu, stan początkowy oraz sposób jego rozwiązania będą następować zaraz po sobie. Schemat ten zostanie powtórzony dla każdego zagadnienia poruszanego w ramach pracy. Autorzy chcą w ten sposób ułatwić użycie tegoż dokumentu biorąc pod uwagę, iż manuskrypt ma być stosowany zarówno jako wprowadzenie, jak i dokumentacja projektu w celu dalszego rozwoju.


Ostatnim z celów postwionych autorom było odpowiednie udokumentowanie projektu tak, aby ewentualne przyszłe zmiany można było wykonywać z jak największą łatwością, a wprowadzenie nowych osób w projekt było jak najprostsze. Oprócz obszernego opisu zawartego w ramach tego manuskryptu wymogiem było, aby przygotować krótkie, lecz treściwe pliki instruktażowe, opisowe oraz odpowiednio udokumentować kod źródłowy.