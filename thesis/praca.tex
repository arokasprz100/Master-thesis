\documentclass[11pt]{aghdpl}

\usepackage[english,polish]{babel}
\usepackage{polski} % Użyj polskiego łamania wyrazów (zamiast domyślnego angielskiego).
\usepackage[utf8]{inputenc}

% dodatkowe pakiety
\usepackage{afterpage}
\usepackage{mathtools}
\usepackage{amsfonts}
\usepackage{amsmath}
\usepackage{amsthm}
\usepackage{float} 
\usepackage{realboxes}
\usepackage{xpatch}
\usepackage{lscape} % landscape
\usepackage{pdfpages} % Do dodawnaie stron pdf jako część dokumentu (nie lstlisting)
\usepackage{emptypage} % Paczka wyłącza pokazywanie numeru strony oraz nagłówków na pustych stronach
\usepackage{siunitx}

% Dzięki temu będzie się dało kopiować tekst który ma polskie literki
% see: https://tex.stackexchange.com/questions/57915/cannot-copy-letters-with-diacritics-from-pdflatex-pdf
\usepackage{lmodern}
\usepackage{listings}
\usepackage{multicol}

% Polskie literki w listingach:
\lstset{
    literate=
    {ą}{{\k{a}}}1 {Ą}{{\k{A}}}1 
    {ł}{{\l{}}}1 {Ł}{{\L{}}}1 
    {ń}{{\'n}}1 {Ń}{{\'N}}1 
    {ę}{{\k{e}}}1 {Ę}{{\k{E}}}1 
    {ś}{{\'s}}1 {Ś}{{\'S}}1 
    {ż}{{\.z}}1 {Ż}{{\.Z}}1 
    {ó}{{\'o}}1 {Ó}{{\'O}}1 
    {ź}{{\'z}}1 {Ź}{{\'Z}}1 
    {ć}{{\'c}}1 {Ć}{{\'C}}1
}


% Definicja listingu dla Dockerfile
% https://gordonlesti.com/custom-code-highlighting-in-latex/
\lstdefinelanguage{Dockerfile}
{
  morekeywords={FROM, RUN, CMD, LABEL, MAINTAINER, EXPOSE, ENV, ADD, COPY,
    ENTRYPOINT, VOLUME, USER, WORKDIR, ARG, ONBUILD, STOPSIGNAL, HEALTHCHECK,
    SHELL},
  morecomment=[l]{\#},
  morestring=[b]"
}

% Definicja listingu dla CMAKE
\lstdefinelanguage{cmake}
{
  morekeywords={set, cmake_minimum_required, project, add_executable, target_include_directories, include, target_link_libraries, file, add_library, set_target_properties, if, endif, list, foreach, endforeach, add_subdirectory, get_directory_property, get_filename_component, function, endfunction, execute_process, ExternalProject_Add, target_link_directories, message, else},
  morecomment=[l]{\#},
  morestring=[b]"
}

\lstdefinelanguage{Cmd}
{
    moredelim=[s][\color{red}\bfseries]{user@host}{\$},
    morekeywords={}
}

\definecolor{darkgreenforcomments}{rgb}{0.0,0.26,0.15}

\lstdefinelanguage{yaml}
{
  keywords={true,false,null,y,n},
  sensitive=false,
  comment=[l]{\#},
  morecomment=[s]{/*}{*/},
  commentstyle=\color{darkgreenforcomments}\ttfamily,
  stringstyle={\color{blue}\mdseries},
  moredelim=[l][\color{orange}]{\&},
  moredelim=[l][\color{magenta}]{*},
  moredelim=**[il][\color{red}{:}\color{blue}]{:},
  morestring=[b]',
  morestring=[b]"
}

\definecolor{lgray}{gray}{0.96}
\definecolor{lbcolor}{rgb}{0.9,0.9,0.9}
\lstset{
    framesep=2pt,
    basicstyle=\ttfamily,
    breaklines=true,
    breakatwhitespace=true,
    basicstyle=\footnotesize,
    aboveskip={0.75\baselineskip},
    columns=fixed,
    showstringspaces=false,
    breaklines=true,
    prebreak = \raisebox{0ex}[0ex][0ex]{\ensuremath{\hookleftarrow}},
    frame=single,
    rulecolor=\color{lgray},
    showtabs=false,
    showspaces=false,
    showstringspaces=false,
    backgroundcolor=\color{lgray},
    identifierstyle=\ttfamily,
    keywordstyle=\color[rgb]{0,0,1},
    commentstyle=\color[rgb]{0.0,0.26,0.15},
    stringstyle=\color[rgb]{0.627,0.126,0.941}
}

% Padding w podpisach do listingow i obrazkow
% ustawiony na 0
\captionsetup[lstlisting]{ margin=0pt}
\captionsetup[figure]{ margin=0pt}
\captionsetup[table]{ margin=0pt}



\makeatletter
\xpretocmd\lstinline{\Colorbox{lgray}\bgroup\appto\lst@DeInit{\egroup}}{}{}
\makeatother

\usepackage[hidelinks]{hyperref}

% --- < bibliografia > ---
%
% TODO: Dla TeXstudio warto przeczytać poniższe!
% UWAGA: Żeby bibliografia działała gdy używamy TeXstudio, to należy zmodyfikować quick build w opcjach, tak aby wykonywane były polecenia:
% 	PdfLaTeX
% 	BibTeX
% 	PdfLaTeX
% 	PdfLaTeX
%
% Więcej informacji na:
% 	https://tex.stackexchange.com/a/216325
%

\usepackage[
	backend=bibtex,
	style=numeric,
	sorting=none,
	% Zastosuj styl wpisu bibliograficznego właściwy językowi publikacji.
	language=autobib,
	autolang=other,
	% Zapisuj datę dostępu do strony WWW w formacie RRRR-MM-DD.
	urldate=iso8601,
	% Nie dodawaj numerów stron, na których występuje cytowanie.
	backref=false,
	% Podawaj ISBN.
	isbn=true,
	% Nie podawaj URL-i, o ile nie jest to konieczne.
	url=false,
	% Ustawienia związane z polskimi normami dla bibliografii.
	maxnames=50,
]{biblatex}

\usepackage{csquotes}
% Ponieważ `csquotes` nie posiada polskiego stylu, można skorzystać z mocno zbliżonego stylu chorwackiego.
\DeclareQuoteAlias{croatian}{polish}

% Bibliografia musi być uzupełniona w tym pliku:
\addbibresource{bibliografia.bib}

% Nie wyświetlaj wybranych pól.
%\AtEveryBibitem{\clearfield{note}}


% ------------------------
% --- < listingi > ---

% Użyj czcionki kroju Courier.
\usepackage{courier}

\lstloadlanguages{TeX}

% Polskie literki:
\lstset{
	literate={ą}{{\k{a}}}1
           {ć}{{\'c}}1
           {ę}{{\k{e}}}1
           {ó}{{\'o}}1
           {ń}{{\'n}}1
           {ł}{{\l{}}}1
           {ś}{{\'s}}1
           {ź}{{\'z}}1
           {ż}{{\.z}}1
           {Ą}{{\k{A}}}1
           {Ć}{{\'C}}1
           {Ę}{{\k{E}}}1
           {Ó}{{\'O}}1
           {Ń}{{\'N}}1
           {Ł}{{\L{}}}1
           {Ś}{{\'S}}1
           {Ź}{{\'Z}}1
           {Ż}{{\.Z}}1,
	basicstyle=\footnotesize\ttfamily,
}

% ------------------------
% Określamy nazwy 'table' oraz 'figure':
\AtBeginDocument{
	\renewcommand{\tablename}{Tabela}
	\renewcommand{\figurename}{Rys.}
}

% ------------------------
% --- < tabele > ---

\usepackage{tabularx}
\usepackage{multirow}
\usepackage{booktabs}
\usepackage{makecell}
\usepackage{float}

% Brak odstępów między wypunktowaniami
\usepackage[inline]{enumitem}
\setlist{nosep}

\setlength{\cftsecnumwidth}{10mm}

%---------------------------------------------------------------------------
\setcounter{secnumdepth}{4}

\begin{document}

%%%%%%%%%%%%%%%%%%%%%%%%%%%%%%%%%%%%%%%%%%%%%%%%%%%%%%%%%%%%%%%%%%%%%%%%%%%%%%%%%%%%%%
%%%%%%%%%%%%%%%%%%%%%%%%%%%%%%%%%%%%%%%%%%%%%%%%%%%%%%%%%%%%%%%%%%%%%%%%%%%%%%%%%%%%%%
%%%%%%%%%%%%%%%%%%%%%%%%%%%%%%%%%%%%%%%%%%%%%%%%%%%%%%%%%%%%%%%%%%%%%%%%%%%%%%%%%%%%%%
%%%%%%%%%%%%%%%%%%%%%%%%%%%%%%%%%%%%%%%%%%%%%%%%%%%%%%%%%%%%%%%%%%%%%%%%%%%%%%%%%%%%%%

\thispagestyle{empty}
\begin{center}
\includegraphics[scale=1.4]{res/agh_nzw_s_pl_1w_wbr_cmyk.pdf} \\[0.2cm]

WYDZIAŁ FIZYKI I INFORMATYKI STOSOWANEJ \\[0.2cm]
KATEDRA ODDZIAŁYWAŃ I DETEKCJI CZĄSTEK \\[1.2cm]
\textbf{\huge Praca Dyplomowa} \\[1.2cm]


%% ------------------------ TYTUL PRACY --------------------------------------

{\LARGE Rozbudowa i uaktualnienie systemu GGSS detektora ATLAS TRT}\\[0.8cm]
{\LARGE Update and upgrade of the GGSS system for ATLAS TRT detector}\\

\vfill
%% ------------------------ DANE PRACY ------------------------------------

\begin{minipage}{\textwidth}
\begin{flushleft}
{
    \large 
    Autorzy: \hfill \textbf{Arkadiusz Kasprzak, Jarosław Cierpich} \\[0.1cm]
    Kierunek studiów: \hfill \textbf{Informatyka Stosowana} \\[0.1cm]
    Opiekun pracy: \hfill \textbf{dr hab. inż. Bartosz Mindur, prof. AGH} \\[0.1cm]
}
\end{flushleft}
\end{minipage} \\[2cm]


{\large \bf \textsf{Kraków, 2021}}
\end{center}

%%%%%%%%%%%%%%%%%%%%%%%%%%%%%%%%%%%%%%%%%%%%%%%%%%%%%%%%%%%%%%%%%%%%%%%%%%%%%%%%%%%%%%
%%%%%%%%%%%%%%%%%%%%%%%%%%%%%%%%%%%%%%%%%%%%%%%%%%%%%%%%%%%%%%%%%%%%%%%%%%%%%%%%%%%%%%
%%%%%%%%%%%%%%%%%%%%%%%%%%%%%%%%%%%%%%%%%%%%%%%%%%%%%%%%%%%%%%%%%%%%%%%%%%%%%%%%%%%%%%
%%%%%%%%%%%%%%%%%%%%%%%%%%%%%%%%%%%%%%%%%%%%%%%%%%%%%%%%%%%%%%%%%%%%%%%%%%%%%%%%%%%%%%

\newpage
\begin{center}
        {\bf\large\textsf{Oświadczenie studenta}}
\end{center}

{\sf Uprzedzony(-a) o odpowiedzialności karnej na podstawie art. 115 ust. 1 i 2 ustawy z dnia 4 lutego 1994 r. o prawie autorskim i prawach pokrewnych (t.j. Dz. U. z 2018 r. poz. 1191 z późn. zm.): ,,Kto przywłaszcza sobie autorstwo albo wprowadza w błąd co do autorstwa całości lub części cudzego utworu albo artystycznego wykonania, podlega grzywnie, karze ograniczenia wolności albo pozbawienia wolności do lat 3. Tej samej karze podlega, kto rozpowszechnia bez podania nazwiska lub pseudonimu twórcy cudzy utwór w wersji oryginalnej albo w postaci opracowania, artystyczne wykonanie albo publicznie zniekształca taki utwór, artystyczne wykonanie, fonogram, wideogram lub nadanie.'', a także uprzedzony(-a) o odpowiedzialności dyscyplinarnej na podstawie art. 307 ust. 1 ustawy z dnia 20 lipca 2018 r. Prawo o szkolnictwie wyższym i nauce (Dz. U. z 2018 r. poz. 1668 z późn. zm.) ,,Student podlega odpowiedzialności dyscyplinarnej za naruszenie przepisów obowiązujących w~uczelni oraz za czyn uchybiający godności studenta.'', oświadczam, że niniejszą pracę dyplomową wykonałem(-am) osobiście i samodzielnie i nie korzystałem(-am) ze źródeł innych niż wymienione w pracy.

\bigskip

Jednocześnie Uczelnia informuje, że zgodnie z art. 15a ww. ustawy o prawie autorskim i prawach pokrewnych Uczelni przysługuje pierwszeństwo w opublikowaniu pracy dyplomowej studenta. Jeżeli Uczelnia nie opublikowała pracy dyplomowej w terminie 6 miesięcy od dnia jej obrony, autor może ją opublikować, chyba że praca jest częścią utworu zbiorowego. Ponadto Uczelnia jako podmiot, o którym mowa w art. 7 ust. 1 pkt 1 ustawy z dnia 20 lipca 2018 r. --- Prawo o szkolnictwie wyższym i nauce (Dz. U. z 2018 r. poz. 1668 z późn. zm.), może korzystać bez wynagrodzenia i bez konieczności uzyskania zgody autora z utworu stworzonego przez studenta w wyniku wykonywania obowiązków związanych z odbywaniem studiów, udostępniać utwór ministrowi właściwemu do spraw szkolnictwa wyższego i~nauki oraz korzystać z utworów znajdujących się w prowadzonych przez niego bazach danych, w celu sprawdzania z wykorzystaniem systemu antyplagiatowego. Minister właściwy do spraw szkolnictwa wyższego i nauki może korzystać z prac dyplomowych znajdujących się w prowadzonych przez niego bazach danych w zakresie niezbędnym do zapewnienia prawidłowego utrzymania i rozwoju tych baz oraz współpracujących z nimi systemów informatycznych.}

\vspace{14ex}

\begin{center}
\begin{tabular}{lr}
~~~~~~~~~~~~~~~~~~~~~~~~~~~~~~~~~~~~~~~~~~~~~~~~~~~~~~~~~~~~~~~~~ &
................................................................. \\
~ & {\sf (czytelny podpis)}\\
\end{tabular}
\end{center}

%%%%%%%%%%%%%%%%%%%%%%%%%%%%%%%%%%%%%%%%%%%%%%%%%%%%%%%%%%%%%%%%%%%%%%%%%%%%%%%%%%%%%%
%%%%%%%%%%%%%%%%%%%%%%%%%%%%%%%%%%%%%%%%%%%%%%%%%%%%%%%%%%%%%%%%%%%%%%%%%%%%%%%%%%%%%%
%%%%%%%%%%%%%%%%%%%%%%%%%%%%%%%%%%%%%%%%%%%%%%%%%%%%%%%%%%%%%%%%%%%%%%%%%%%%%%%%%%%%%%
%%%%%%%%%%%%%%%%%%%%%%%%%%%%%%%%%%%%%%%%%%%%%%%%%%%%%%%%%%%%%%%%%%%%%%%%%%%%%%%%%%%%%%

\newpage
\rightline{Kraków, ?? września 2021}
\begin{center}
{\bf Tematyka pracy magisterskiej i praktyki dyplomowej
Jarosława Cierpicha,
studenta drugiego roku studiów drugiego stopnia na kierunku informatyka stosowana, specjalności modelowanie i analiza danych}\\
\end{center}

Temat pracy magisterskiej:
{\bf Rozbudowa i uaktualnienie systemu GGSS detektora ATLAS TRT}\\

\begin{tabular}{rl}

Opiekun pracy:                  & dr hab. inż. Bartosz Mindur, prof. AGH \\
Recenzenci pracy:               & \\
Miejsce praktyki dyplomowej:    & WFiIS AGH, Kraków\\
\end{tabular}

\begin{center}
{\bf Program pracy magisterskiej i praktyki dyplomowej}
\end{center}

\begin{enumerate}
\item Omówienie realizacji pracy magisterskiej z opiekunem.
\item Zebranie i opracowanie literatury dotyczącej tematu pracy.
\item Praktyka dyplomowa:
\begin{itemize}
    \item udział w Krakow Applied Physics and Computer Science Summer School '20
    \item zapoznanie z materiałami (wykłady i szkolenia praktyczne) obejmującymi zagadnienia z dziedziny fizyki cząstek, informatyki oraz detektorów i elektroniki
    \item praca nad projektem GGSS w dwuosobowym zespole, obejmująca zmiany w oprogramowaniu i architekturze projektu
    \item prezentacja rezultatów wykonanej pracy przed uczestnikami oraz opiekunami szkoły
    \item prezentacja wykonanych prac podczas wydarzenia TRT Days
\end{itemize}
\item Kontynuacja prac nad projektem:
\begin{itemize}
    \item wykonanie dalszych zmian w oprogramowaniu systemu GGSS, w tym dodanie nowych funkcjonalności
    \item przeprowadzanie okresowych testów działania systemu w środowisku docelowym
    \item wykonanie prac nad infrastrukturą projektu
\end{itemize}
\item Opracowanie redakcyjne pracy.
\end{enumerate}


\noindent
Termin oddania w dziekanacie: ?? września 2021\\[1cm]

\begin{center}
\begin{tabular}{lcr}
.............................................................. & ~~~ &
.............................................................. \\
(podpis kierownika katedry) & & (podpis opiekuna) \\
\end{tabular}
\end{center}

%%%%%%%%%%%%%%%%%%%%%%%%%%%%%%%%%%%%%%%%%%%%%%%%%%%%%%%%%%%%%%%%%%%%%%%%%%%%%%%%%%%%%%
%%%%%%%%%%%%%%%%%%%%%%%%%%%%%%%%%%%%%%%%%%%%%%%%%%%%%%%%%%%%%%%%%%%%%%%%%%%%%%%%%%%%%%
%%%%%%%%%%%%%%%%%%%%%%%%%%%%%%%%%%%%%%%%%%%%%%%%%%%%%%%%%%%%%%%%%%%%%%%%%%%%%%%%%%%%%%
%%%%%%%%%%%%%%%%%%%%%%%%%%%%%%%%%%%%%%%%%%%%%%%%%%%%%%%%%%%%%%%%%%%%%%%%%%%%%%%%%%%%%%

\newpage
\mbox{} % The empty page

%%%%%%%%%%%%%%%%%%%%%%%%%%%%%%%%%%%%%%%%%%%%%%%%%%%%%%%%%%%%%%%%%%%%%%%%%%%%%%%%%%%%%%
%%%%%%%%%%%%%%%%%%%%%%%%%%%%%%%%%%%%%%%%%%%%%%%%%%%%%%%%%%%%%%%%%%%%%%%%%%%%%%%%%%%%%%
%%%%%%%%%%%%%%%%%%%%%%%%%%%%%%%%%%%%%%%%%%%%%%%%%%%%%%%%%%%%%%%%%%%%%%%%%%%%%%%%%%%%%%
%%%%%%%%%%%%%%%%%%%%%%%%%%%%%%%%%%%%%%%%%%%%%%%%%%%%%%%%%%%%%%%%%%%%%%%%%%%%%%%%%%%%%%

\newpage
\rightline{Kraków, ?? września 2021}
\begin{center}
{\bf Tematyka pracy magisterskiej i praktyki dyplomowej Arkadiusza Kasprzaka,
studenta drugiego roku studiów drugiego stopnia na kierunku informatyka stosowana, specjalności modelowanie i analiza danych}\\
\end{center}

Temat pracy magisterskiej:
{\bf Rozbudowa i uaktualnienie systemu GGSS detektora ATLAS TRT}\\

\begin{tabular}{rl}

Opiekun pracy:                  & dr hab. inż. Bartosz Mindur, prof. AGH \\
Recenzenci pracy:               & \\
Miejsce praktyki dyplomowej:    & WFiIS AGH, Kraków\\
\end{tabular}

\begin{center}
{\bf Program pracy magisterskiej i praktyki dyplomowej}
\end{center}

\begin{enumerate}
\item Omówienie realizacji pracy magisterskiej z opiekunem.
\item Zebranie i opracowanie literatury dotyczącej tematu pracy.
\item Praktyka dyplomowa:
\begin{itemize}
    \item udział w Krakow Applied Physics and Computer Science Summer School '20
    \item zapoznanie z materiałami (wykłady i szkolenia praktyczne) obejmującymi zagadnienia z dziedziny fizyki cząstek, informatyki oraz detektorów i elektroniki
    \item praca nad projektem GGSS w dwuosobowym zespole, obejmująca zmiany w oprogramowaniu i architekturze projektu
    \item prezentacja rezultatów wykonanej pracy przed uczestnikami oraz opiekunami szkoły
    \item prezentacja wykonanych prac podczas wydarzenia TRT Days
\end{itemize}
\item Kontynuacja prac nad projektem:
\begin{itemize}
    \item wykonanie dalszych zmian w oprogramowaniu systemu GGSS, w tym dodanie nowych funkcjonalności
    \item przeprowadzanie okresowych testów działania systemu w środowisku docelowym
    \item wykonanie prac nad infrastrukturą projektu
\end{itemize}
\item Opracowanie redakcyjne pracy.
\end{enumerate}


\noindent
Termin oddania w dziekanacie: ?? września 2021\\[1cm]

\begin{center}
\begin{tabular}{lcr}
.............................................................. & ~~~ &
.............................................................. \\
(podpis kierownika katedry) & & (podpis opiekuna) \\
\end{tabular}
\end{center}

%%%%%%%%%%%%%%%%%%%%%%%%%%%%%%%%%%%%%%%%%%%%%%%%%%%%%%%%%%%%%%%%%%%%%%%%%%%%%%%%%%%%%%
%%%%%%%%%%%%%%%%%%%%%%%%%%%%%%%%%%%%%%%%%%%%%%%%%%%%%%%%%%%%%%%%%%%%%%%%%%%%%%%%%%%%%%
%%%%%%%%%%%%%%%%%%%%%%%%%%%%%%%%%%%%%%%%%%%%%%%%%%%%%%%%%%%%%%%%%%%%%%%%%%%%%%%%%%%%%%
%%%%%%%%%%%%%%%%%%%%%%%%%%%%%%%%%%%%%%%%%%%%%%%%%%%%%%%%%%%%%%%%%%%%%%%%%%%%%%%%%%%%%%


% Ponowne zdefiniowanie stylu `plain`, aby usunąć numer strony z pierwszej strony spisu treści i poszczególnych rozdziałów.
\fancypagestyle{plain}
{
	% Usuń nagłówek i stopkę
	\fancyhf{}
	% Usuń linie.
	\renewcommand{\headrulewidth}{0pt}
	\renewcommand{\footrulewidth}{0pt}
} 


% Recenzja nr 1
% \includepdfset{pages=-,pagecommand=\thispagestyle{fancy}}
% \includepdf[pages=1, scale=0.9,offset= 0.65cm -1.2cm, pagecommand={}]{res/rec1_op.pdf}

\clearpage\mbox{}\clearpage

% Recenzja nr 2
% \includepdfset{pages=-,pagecommand=\thispagestyle{fancy}}
% \includepdf[pages=1, scale=0.9,offset= 0.65cm -1.2cm, pagecommand={}]{res/rec1_or.pdf}

\clearpage\mbox{}\clearpage

\clearpage
\setcounter{tocdepth}{2}
\tableofcontents
\clearpage

\chapter{Wstęp}
\label{cha:wstep}

\section{Wprowadzenie do systemu GGSS}

Europejska Organizacja Badań Jądrowych CERN jest jednym z najważniejszych ośrodków naukowo-badawczych na świecie i miejscem rozwoju zarówno fizyki, jak i informatyki. Będąc miejscem powstania wielu znaczących technologii (m.in. protokół \emph{HTTP} - \emph{Hypertext Transfer Protocol}), CERN kojarzony jest dziś przede wszystkim z Wielkim Zderzaczem Hadronów (\emph{LHC} - \emph{Large Hadron Collider}) - największym akceleratorem cząstek na świecie. Jednym z pracujących przy LHC eksperymentów jest detektor ATLAS (\emph{A Toroidal LHC ApparatuS}), pełniący kluczową rolę w rozwoju współczesnej fizyki - przyczynił się on do potwierdzenia istnienia tzw. bozonu Higgsa w 2012 roku. 

Detektor ATLAS zbudowany jest z kilku pod-detektorów, tworzących strukturę warstową. Najbardziej wewnętrzną część stanowi tzw. Detektor Wewnętrzny (ang. \emph{Inner Detector}), składający się z kolei z trzech kolejnych podsystemów. Jednym z tychże podsystemów, szczególnie istotnym w kontekście niniejszej pracy, jest detektor promieniowania przejścia (\emph{TRT} - \emph{Transition Radiation Tracker}). 

System Stabilizacji Wzmocnienia Gazowego (\emph{GGSS} - \emph{Gas Gain Stabilization System}) jest jednym z podsystemów detektora TRT, mającym zapewnić jego poprawne działanie. Projekt ten zintegrowany jest z systemem kontroli detektora ATLAS (\emph{DCS} - \emph{Detector Control System}). W skład systemu GGSS wchodzi zarówno warstwa oprogramowania, jak i szereg urządzeń. Ze względu na jego rolę, jednym z najważniejszych wymagań stawianych przed projektem jest wysoka niezawodność.

W niniejszej pracy autorzy przybliżą najważniejsze zmiany dokonane przez nich w czasie półtorarocznych prac nad rozwojem i usprawnieniem systemu GGSS. Prace obejmują przede wszystkim zmiany w warstwie oprogramowania, mające na celu zarówno wprowadzenia nowych funkcjonalności do systemu, jak również uczynienie go bardziej przystępnym dla korzystających z niego osób, m.in. poprzez automatyzację procesów związanych z cyklem życia oprogramowania (np. tworzenie nowych wydań).


\section{Cel pracy}

\chapter{Budowa i działanie systemu GGSS}
\label{cha:ggss}

Niniejszy rozdział zawiera ważne, z punktu widzenia przeprowadzonych prac, informacje na temat systemu GGSS. Przedstawione tu opisy dotyczą zagadnień takich jak: wysokopoziomowa architektura systemu, struktura warstwy oprogramowania, opis wykorzystywanych przez system urządzeń oraz omówienie cech charakterystycznych środowiska docelowego. 


\section{Wysokopoziomowa architektura systemu GGSS}
System GGSS składa się z kilku współpracujących ze sobą elementów, przedstawionych (wraz z występującymi między nimi interakcjami) na rysunku \ref{fig:high_level_architecture}. Znaczenie poszczególnych komponentów projektu jest następujące:
\begin{itemize}
    \item \textbf{oprogramowanie GGSS} - zestaw aplikacji wraz z otaczającą je infrastrukturą, których zadaniem jest sterowanie urządzeniami wchodzącymi w skład systemu GGSS oraz przetwarzanie zbieranych za ich pomocą danych
    \item \textbf{urządzenia (ang. \textit{hardware})} - zestaw urządzeń elektronicznych (m.in. liczniki słomkowe, zasilacze wysokiego napięcia i multipleksery)
    \item \textbf{pliki konfiguracyjne} - proste pliki tekstowe w formacie XML \textbf{rozwin + cytowanie}, zawierające informacje o oczekiwanym sposobie działania systemu (np. maksymalna możliwa wartość napięcia, jakie może zostać ustawione na każdym z zasilaczy)
    \item \textbf{pliki wynikowe} - pliki tekstowe zawierające wyniki pomiarów wykonywanych przez system oraz rejestr zdarzeń 
    \item \textbf{system WinCC OA} - \textbf{rozwin + cytowanie} - system typu SCADA \textbf{rozwin + cytowanie}, stanowiący część systemu kontroli detektora ATLAS (DCS), pozwalający na obserwację i kontrolę działania poszczególnych poddetektorów
    \item \textbf{protokół DIM} - \textbf{rozwin + cytowanie} protokół komunikacyjny dla środowisk rozproszonych, oparty o architekturę klient-serwer, zapewniający komunikację między oprogramowaniem systemu GGSS a systemem WinCC OA
\end{itemize}

\begin{figure}[H]
\centering
\includegraphics[width=\textwidth]{components/ggss_images/high_level_architecture.pdf}
\caption{Wysokopoziomowa architektura projektu GGSS. Strzałkami oznaczono przepływ danych pomiędzy poszczególnymi komponentami systemu.}
\label{fig:high_level_architecture}
\end{figure}


Szczegóły działania najważniejszych z punktu widzenia niniejszej pracy elementów systemu omówione zostaną w dalszej części tego rozdziału. Znaczna część prac opisanych w niniejszym manuskrypcie skupiona była na udoskonaleniu warstwy oprogramowania systemu GGSS.


\section{Urządzenia elektroniczne}
Z punktu widzenia warstwy sprzętowej system GGSS składa się z zestawu tzw. słomkowych liczników proporcjonalnych, zasilanych za pomocą zasilaczy wysokiego napięcia. Sygnały generowane przez liczniki przetwarzane są przez wielokanałowy analizator amplitudy (MCA - \textbf{rozwinac}), natomiast wybór licznika słomkowego używanego do wykonania pomiarów następuje za pomocą multipleksera analogowego \textbf{cytowanie}. Urządzenia podłączone są do komputera PC, który steruje nimi za pomocą oprogramowania systemu GGSS. W tabeli \ref{tab:devices} zamieszczone zostało zestawienie informacji na temat wykorzystywanych przez projekt urządzeń. Sposób działania systemu (jego podstawa fizyczna oraz znaczenie przeprowadzanych pomiarów) wykracza poza zakres niniejszej pracy, został natomiast szczegółowo opisany w pracy \emph{Wybrane zagadnienia związane z pracą słomkowych liczników proporcjonalnych w detektorze TRT eksperymentu ATLAS}, której autorem jest dr hab. inż. Bartosz Mindur, prof. AGH \textbf{cytowanie}.

\clearpage

\begin{table*}[htbp]
\centering
\caption{Zestawienie istotnych z punktu widzenia niniejszej pracy urządzeń wchodzących w skład systemu GGSS.}
\label{tab:devices}
\begin{tabularx}{\textwidth}{@{}XX@{}}
\toprule
Urządzenie &
Informacje \\
\midrule
4-kanałowy zasilacz wysokiego napięcia & CAEN N1470 \\
wielokanałowy analizator amplitudy & CAEN N957 \\
multiplekser sygnałów analogowych & urządzenie autorstwa Pana Pawła Zadrożniaka\\
\bottomrule
\end{tabularx}
\end{table*}


\section{Warstwa oprogramowania}
Poprzez warstwę oprogramowania systemu GGSS autorzy rozumieją zarówno zestaw aplikacji napisanych w języku C++ (standard 11), jak i otaczającą je infrastrukturę (pomocnicze skrypty, system budowania, testowania i tworzenia nowych wydań). 



\subsection{Aplikacje wchodzące w skład GGSS}
Trzon warstwy oprogramowania projektu GGSS stanowi aplikacja \emph{ggss-runner}, zawierająca logikę odpowiedzialną za komunikację z systemem za pomocą protokołu DIM, gromadzenie i walidację danych oraz sterowanie urządzeniami wchodzącymi w skład warstwy sprzętowej. W skład systemu wchodzi ponadto szereg pomniejszych aplikacji (niektóre z nich stanowią element dodany przez autorów niniejszej pracy, zostaną więc omówione ze szczegółami w dalszych jej częściach):
\begin{itemize}
    \item \emph{ggss-spector} - aplikacja okienkowa służąca do wizualizacji zebranych przez system danych (zapisanych w plikach wynikowych)
    \item \emph{ggss-reader} - niezależna aplikacja przeznaczona do wykorzystywania na maszynach deweloperskich, pozwalająca na odtwarzanie działania oprogramowania sterującego GGSS, tzn. wysyłająca do systemu kontroli detektora archiwalne dane z pominięciem warstwy sprzętowej \textbf{cytowanie pracy grzeska}
    \item \emph{ggss-dim-cs} - aplikacja pozwalająca na wysyłanie do systemu komend za pomocą protokołu DIM \textbf{tutaj cytowanie}
    \item zestaw aplikacji \emph{ggss-hardware-service-apps} - proste narzędzia pozwalające na wykonywanie operacji na wchodzących w skład systemu urządzaniach, w tym na wykonywanie testów ich działania. 
\end{itemize}


\subsection{Język C++}
Zarówno aplikacja \emph{ggss-runner}, jak i wszystkie aplikacje pomocnicze, napisane zostały za pomocą języka C++. Jest to wydajny, wszechstronny język programowania ogólnego przeznaczenia, pozwalający programiście zarówno na wykorzystywanie wysokopoziomowych abstrakcji (programowanie obiektowe, generyczne i funkcyjne), jak i na wydajne wykonywanie niskopoziomowych operacji. W ciągu ostatnich dziesięciu lat język ten był intensywnie rozwijany - od 2011 roku pojawiły się cztery nowe standardy, w tym najnowszy w roku 2020, a kolejny przewidziany jest na rok 2023. Zmiany wprowadzane w nowych wydaniach języka mają na celu zarówno dodawanie do niego nowych funkcjonalności, jak równeż promowanie praktych pozwalających tworzyć prosty w utrzymaniu, czytelny kod. Niestety z uwagi na ograniczenia wynikające ze cech środowiska docelowego, w jakim działać ma system GGSS, w omawianym projekcie możliwe było wykorzystanie jedynie standardu C++11. 


\subsection{Biblioteki zewnętrzne}
W projekcie wykorzystywane są ponadto biblioteki nie będące częścią standardu języka C++, dostarczające funkcjonalności niezbędnych do poprawnego działania systemu. Najważniejsze z nich to:
\begin{itemize}
    \item \emph{Boost} \textbf{cytowanie} - rozbudowany zestaw bibliotek dla języka C++, cieszący się znaczącą popularnością, m.in. ze względu na wysoką jakość i szeroki zakres wprowadzanych funkcjonalności (m.in. przetwarzanie argumentów linii poleceń, implementacja operacji na grafach, wsparcie dla programowania sieciowego, tworzenia testów jednostkowych czy zaawansowanego metaprogramowania). Ponadto niektóre z bibliotek wchodzących w skład \emph{Boost} były podstawą do implementacji funkcjonalności takich jak inteligentne wskaźniki (ang. smart pointers) czy obsługa wyrażeń regularnych w nowych standardach języka C++.
    \item \emph{GNU Scientific Library (GSL)} \textbf{cytowanie} - biblioteka dla języków C i C++, dostarczająca implementacje popularnych algorytmów numerycznych
    \item \emph{Qt} oraz \emph{Qwt} \textbf{cytowanie} - wieloplatformowy zestaw bibliotek i narzędzi pozwalających na tworzenie aplikacji okienkowych, wykorzystywany przez aplikacje \emph{ggss-spector} oraz \emph{ggss-reader}
    \item \emph{DIM} \textbf{cytowanie} - dostarczona przez CERN biblioteka umożliwiająca wykorzystywanie protokołu DIM
    \item \emph{CAEN-N957} \textbf{cytowanie} - dostarczona przez firmę CAEN biblioteka współdzielona napisana w języku C, służąca do obsługi analizatora wielokanałowego N957
\end{itemize}


\subsection{Infrastruktura projektu}
Projekt GGSS charakteryzuje się rozbudowaną infrastrukturą, w której skład wchodzą systemy odpowiedzialne za budowanie projektu, zarządzanie zależnościami pomiędzy jego komponentami (również wewnątrz samego projektu), automatyzację procesu testowania poszczególnych komponentów oraz automatyzację tworzenia i wersjonowania wydań. Ponadto w jej skład wchodzą skrypty pomocnicze (napisane w językach Bash \textbf{cytat} i Python \textbf{cytat}), pozwalające na zarządzanie systemem w jego środowisku docelowym. Gruntowna przebudowa infrastruktury systemu GGSS stanowiła temat pracy inżynierskiej autorów (\textbf{cytat i tytul}). W dalszej części niniejszego manuskryptu omówione zostaną wprowadzone w ramach pracy magistersiej rozszerzenia. 

Infrastruktura budowania projektu oparta jest o narzędzie CMake (\textbf{cytowanie}) (wersja 2.8) rozwijane przez firmę \emph{Kitware}. Zastosowanie go pozwala na generowanie pliku budującego projekt właściwego dla danej platformy docelowej (np. \emph{Makefile} dla systemów z rodziny UNIX), a co za tym idzie, pozwala na łatwe tworzenie aplikacji wieloplatformowych. Stosując narzędzie CMake, tworzenie systemu budowania projektu polega na przygotowaniu pliku (lub zestawu plików) \emph{CMakeLists.txt}, zawierającego polecenia pozwalające na określenie przez programistę informacji takich jak: standard języka C++, lokalizacja plików źródłowych i bibliotek zewnętrznych.

\section{Oprogramowanie WinCC OA}
SIMATIC WinCC Open Architecture jest oprogramowaniem typu SCADA firmy SIEMENS służącym do wizualizacji i sterowania procesami produkcyjnymi. System ten stanowi trzon systemu kontroli detektora ATLAS (DCS) i pozwala na monitorowanie i sterowanie pracą wchodzących w jego skład podsystemów. WinCC OA pozwala m.in. na tworzenie specjalnych paneli, obrazujacych zebrane dane oraz procesy zachodzące w monitorowanym systemie - przykład tego typu panelu, obrazujący pracę liczników słomkowych wykorzystywanych przez GGSS, przedstawiony został na rysunku \ref{fig:winccoa_panel_example}.

(https://new.siemens.com/global/en/products/automation/industry-software/automation-software/scada/simatic-wincc-oa/wincc-oa-basic-software.html)


W ramach pracy magisterskiej autorzy nie wykonywali żadnych prac związanych z rozwojem czy utrzymaniem oprogramowania WinCC OA. Jest ono natomiast szczególnie istotne z punktu widzenia testów, jakim musiał być poddawany system GGSS wraz z postępami prac. \textbf{dopisac dlaczego, napisac cos jeszcze moze np o pracy grzeska}

% Krotko co to jest
% Co to sa panele, pokazac przyklad
% Krotko jak dziala DIM, co jest u nas klientem, a co serwerem

\begin{figure}[H]
\centering
\includegraphics[width=\textwidth]{components/ggss_images/winccoa_panel.png}
\caption{}
\label{fig:winccoa_panel_example}
\end{figure}

\section{Środowisko docelowe i ograniczenia}
Charakterystyka środowiska docelowego, w jakim działa system GGSS, jest z punktu widzenia niniejszej pracy bardzo istotna - silny związek projektu z infrastrukturą dostarczaną przez CERN stawia przed autorami niniejszej pracy szereg ograniczeń dotyczących wersji wykorzystywanych narzędzi, jak również wymusza dodatkowe działania w przypadku wykonywania pewnych operacji. Do najważniejszych ograniczeń narzucanych przez środowisko docelowe i specyfikę projektu należą:
\begin{itemize}
    \item dostępna wersje kompilatora języka C++ - w ramach infrastruktury CERN dostępny jest kompilator \emph{g++ (GCC) 4.8.5}. Wersja ta wspiera w większości standard C++11, a zatem funkcjonalności takie jak wyrażenia lambda czy semantyka przenoszenia. Niestety oferowane przez nią wsparcie nie jest pełne - brakuje m.in. poprawnej implementacji biblioteki odpowiedzialnej za przetwarzanie wyrażeń regularnych. Ze względu na wymóg zapewnienia możliwości budowania projektu na maszynie docelowej, ograniczenie to stanowiło znaczące utrudnienie podczas prac nad kodem źródłowym aplikacji wchodzących w skład systemu.
    \item dostępna wersja narzędzia CMake - na maszynach docelowych dostępna jest wersja \emph{2.8.12.2}, stanowiąca bardzo stare wydanie narzędzia. \textbf{dopisac cos}
    \item związek projektu z wersją jądra systemu \textbf{dopisac cos}
    \item możliwość przeprowadzania testów tylko w określonych momentach prac nad projektem \textbf{dopisac cos}
    \item ograniczone uprawnienia w środowisku docelowym \textbf{dopisac cos}
    \item konieczność zachowania kompatybilności wstecznej \textbf{dopisac cos}
\end{itemize}



% Dedykowany komputer
% Jakie ograniczenia wynikaja z tego
% Wymog wstecznej kompatybilnosci

\appendix
\chapter{Przegląd praktyk stosowanych podczas prac nad projektem (JC)}
\label{cha:practices}

Celem niniejszego dodatku jest przedstawienie najważniejszych praktyk stosowanych przez autorów podczas wykonywania prac nad systemem GGSS. Poruszone zostały tematy organizacji pracy nad kodem w~zespole, dokumentacja projektu, czy też konwencje zastosowane w~celu uzyskania w~całym projekcie jednolitego kodu źródłowego.

\section{Wprowadzenie do problematyki}
Ze względu na zespołowy charakter przygotowanej przez autorów pracy inżynierskiej, w~trakcie jej wykonywania wprowadzone zostały praktyki mające na celu organizację i~koordynację współpracy. W~ramach platformy GitLab, wykorzystywanej przez CERN jako główne narzędzie do współpracy nad kodem, skorzystano z~szeregu funkcjonalności ułatwiających śledzenie postępów, jak i~zarządzanie projektem. Oprócz utworzenia zespołu, do którego został przypisany kod projektu oraz w~ramach którego odbywała się kolaboracja, wykorzystano:
\begin{itemize}
\item \emph{issue} - opis pojedynczego zadania/problemu. Zawiera podstawowe informacje, przypisaną osobę, etykietę, która oznacza obecny stan, termin wykonania oraz wagę
\item \emph{kanban board} - tablica kanban zawierająca wszystkie przypisane do projektu zadania. Kolumny takiej tabeli stanowią spersonalizowane do projektu etykiety. Pozwala na wysokopoziomowe zarządzanie projektem, sprawdzenie statusu, czy też łatwą zmianę etykiety przypisanej do zadań poprzez przeciągnięcie do odpowiedniej kolumny. % czy pokazac kanban board
\item \emph{merge request} - dedykowany widok do wprowadzania zmian wprowadzonych w~ramach kodu deweloperskiego do kodu produkcyjnego, który jest wykorzystywany do tworzenia i~dostarczania aplikacji
\item \emph{milestone} - kamień milowy, jednostka organizacyjna pozwalająca na grupowanie kilku zadań, które realizują większy cel. \emph{Milestone} śledzi przypisane do niego zadania, przewidywany czas zakończenia oraz wagę pozostałych do wykonania zadań.
\end{itemize}
W~trakcie wykonywania pracy inżynierskiej, szczególnie podczas początkowego etapu projektu, który odbywał się w~trakcie 3-tygodniowego wyjazdu do CERN, wyżej wymienione praktyki sprawdzały się bardzo dobrze.

\section{Motywacja do wprowadzenia zmian}
Ze względu na dobre sprawowanie się wyżej wymienionych praktyk w~trakcie pracy inżynierskiej postanowiono o~kontynuowaniu ich wykorzystania również w~trakcie pracy magisterskiej. Z~powodu nieregularnego aspektu pracy nad projektem, wykonywanie czynności zdalnie, lepsze poznanie środowiska, wykorzystywanych narzędzi oraz samej platformy GitLab postanowiono dostosować stosowane praktyki do nowych realiów. Dodatkowo bardzo ważną zasadą, biorąc pod uwagę zakończenie pracy nad projektem i~przekazanie go osobie odpowiedzialnej za dalsze utrzymanie, było odpowiednie udokumentowanie całego projektu. Wymagało się, aby możliwie proste było wprowadzanie zmian do systemu GGSS oraz sprawne, nieprzerwane działanie po zakończeniu pracy magisterskiej. Dlatego dostarczona dokumentacja musiała być obszerna oraz dobrze opisująca zastosowane rozwiązania. Dodatkowo biorąc pod uwagę mocny nacisk tejże pracy na część aplikacyjną projektu potrzebne było zdefiniowanie pewnych zasad pozwalających na ustandaryzowaną pracę z~kodem źródłowym aplikacji. Pozwoliło to na zachowanie pewnych konwencji w~całym projekcie, co zapobiegało różnicom w~kodzie między komponentami, a~co za tym idzie utrzymanie kodu oraz wdrożenie nowych osób do projektu jest znacznie uproszczone.


\section{Zmiana praktyk ze względu na nieregularność prac}

Prace nad systemem GGSS były kontynuowane, z~mniejszymi przerwami, od obrony pracy inżynierskiej. Natomiast ich charakter był nieregularny. Każdy z~autorów pracował nad systemem w~wybranych przez siebie godzinach. Ze względu na to wszystkie praktyki, które opierały się o~regularny czas pracy oraz przewidywanie czasu zakończenia danych zadań nie miały większego zastosowania. Postanowiono zatem zaprzestać przypisywania wag poszczególnym zadaniom. Oprócz wartości szacunkowej niewiele ona wnosiła w~trakcie wykonywania zadania, dodatkowo zdarzało się, że przybrane wartości różniły się od rzeczywistej wagi problemu, ponieważ często zadania wymagały w~pierwszej kolejności zgłębienia tematu, a~następnie określenia dokładnego rozwiązania problemu.

Zarzucono również praktykę wypełniania pola ,,termin oddania`` w~ramach tworzonych zadań. Ze względu na  wcześniej wspomnianą nieregularność czasu pracy nad projektem, informacja ta często nie zgadzała się z~rzeczywistym czasem zakończenia zadania. Dodatkowo nie była praktycznie w~ogóle potrzebna w~trakcie prac nad projektem ze względu na sposób formułowania zadań, które były możliwe do realizacji bez wpływu na pozostałe problemy. W~przypadku zadań, które wymagały koordynacji, czy też pracy od obydwu autorów, organizowane były spotkania online z~wykorzystaniem narzędzi takich jak Microsoft Teams, które pozwalały na tworzenie konferencji podczas których realizowane były wyżej wymienione zadania, czy też określane były ramy czasowe wykonania zadań od siebie zależnych. Sposób ten sprawdził się bardzo dobrze i~nie wymagana była dodatkowa koordynacja dla tego typu prac.

Rysunek \ref{fig:issue} przedstawia \emph{issue} utworzone według nowo ustalonych zasad. Brak jest przypisanego \emph{milestone}, czy też \emph{due date}. Natomiast ważne, wartościowe informacje, przydatne w~trakcie pracy nad projektem są wypełnione, tj.: rozbudowany opis pozwalający w~krótkim czasie zrozumieć o~co chodzi w~konkretnym zadaniu, osoby przypisane do \emph{issue} oraz etykiety oznaczające aktualny stan wykonania zadania, czy też jakikolwiek powód z~którego \emph{issue} nie zostało, bądź nie zostanie wykonane.

\begin{figure}[H]
    \centering
    \includegraphics[width=\textwidth]{issue.png}
    \caption{Przykładowe \emph{issue} wg. nowo przyjętych praktyk}
    \label{fig:issue}
\end{figure}


\newpage
\section{Dokumentacja projektu}

%Wstepny tekst
Projekt GGSS ma być utrzymywany i~pozostać w~użyciu również po zakończeniu działań nad pracą dyplomową. Ze względu na to że rozwiązania wprowadzone do projektu były zarówno implementowane, jak i~projektowane przez autorów w~porozumieniu z~promotorem, posiadają oni niezbędną wiedzę na temat: powodów zastosowania pewnych rozwiązań, sposobu ich działania, sposobu korzystania z~nich, czy też zasad, które należy stosować w~trakcie rozwoju aplikacji. Ze względu na te czynniki dużo uwagi poświęcono przygotowaniu odpowiedniej dokumentacji pozwalającej na swobodną pracę z~projektem przez osoby, które ten projekt będą nadal utrzymywać.

% Dokumentacja na poziomie README
Dokumentacja w~postaci plików \emph{README} napisanych w~języku znaczników \emph{Markdown} jest dedykowana dla każdego z~repozytoriów. Zazwyczaj opisana jest w~niej zawartość danego repozytorium, sposób użycia tejże zawartości, jeżeli wcześniejsze przygotowanie zawartości jest potrzebne opisane są kroki, które należy w~takiej sytuacji poczynić. Dodatkowo w~wyżej wymienionych plikach opisane są wszelkie niuanse, czy też bardziej zaawansowane kwestie dotyczące zawartości danego repozytorium.

Rysunek \ref{fig:readme} przedstawia przykładowy plik \emph{README} dla repozytorium \emph{ggss-all} zawierającego infrastrukturę do budowy głównej aplikacji systemu GGSS. Wyżej wymieniony plik zawiera informacje o~przeznaczeniu repozytorium, wymaganiach potrzebnych do spełnienia w~celu uruchomienia infrastruktury budującej aplikację, krokach które należy podjąć, aby skorzystać z~tejże infrastruktury. Oprócz tego plik ten zawiera gotowe do użycia komendy, które można skopiować i~wkleić bezpośrednio do konsoli w~celu skorzystania z~infrastruktury. Plik ten zawiera również, a~co nie jest widoczne na rysunku, informacje o~sposobie uzyskania dostępu do kodu protokołu DIM, który jest wymagany do działania systemu GGSS.


Przygotowana w~ten sposób dokumentacja pozwala osobie praktycznie niezapoznanej z~projektem na skorzystanie z~infrastruktury i~przygotowanie gotowej do użycia, w~środowisku docelowym, aplikacji. Również powrót do projektu po dłuższej przerwie nie powinien powodować większych trudności.
\newpage
\begin{figure}[H]
    \centering
    \includegraphics[width=\textwidth]{readme.png}
    \caption{Przykładowe \emph{README} w~ramach repozytorium \emph{ggss-all}}
    \label{fig:readme}
\end{figure} % zembedowac pdf

% Dokumentacja na poziomie kodu źrółowego
W~projekcie została zastosowana również dokumentacja na poziomie kodu źródłowego. Znajduje się ona między innymi: przed klasami, przed metodami, czy też na początku plików źródłowych. Dokumentacja ta stosuje format zgodny z~narzędziem Doxygen, co pozwoliło na jej ujednolicenie i~zwiększenie czytelności. Dzięki wcześniej wspomnianej zgodności możliwe jest wygenerowanie dokumentacji w~postaci plików HTML. Dokumentacja taka, w~celu jej przeczytania, wymaga jedynie aktualnej przeglądarki internetowej. W~celu pełnego wsparcia dokumentacji w~postaci plików HTML generowanych z~użyciem narzędzia Doxygen potrzebne było również dostosowanie infrastruktury służącej do budowania projektu, a~dokładnie plików CMake, dzięki czemu wygenerowane pliki \emph{make} posiadają moduły odpowiedzialne za obsługę wcześniej wspomnianej dokumentacji.

Listing \ref{lst:code_documentation} przedstawia przykładową dokumentację zgodną z~formatem wspieranym przez narzędzie Doxygen. Zawiera ona krótki opis dotyczący metody, następnie opis każdego z~parametrów przyjmowanych przez daną metodą oraz wartość zwracaną przez metodę. Informacje te są bardzo przydatne w~przypadku, gdy programista nie jest pewny, zważając na samą definicję metody, jej działania, parametrów wejściowych, czy też wyjścia. Opis taki rozwiewa częściowo wątpliwości i~pozwala w~poprawny sposób skorzystać z~wcześniej napisanego kodu.

\begin{lstlisting}[
    language=C++,
    caption={Przykładowy fragment kodu biblioteki \emph{fit-lib} wraz z~dokumentacją.},
    label={lst:code_documentation}
]
/**
 * \brief Class with naive peak finding algorithm implementation.
 */
class NaivePeakFinder : public PeakFinder
{
public:

    /**
     * \brief Computes initial peak position for Gauss fit.
     * \param fitData Data used for performing fit and finding peak position.
     * \param fitParams Structure with fit parameters (like fit range).
     * \return Calculated initial peak position.
     */
    double find(const std::vector<double>& fitData,
                const FitParams& fitParams) const override;
};
\end{lstlisting}

Rysunek \ref{fig:doxygen} przedstawia dokumentację jednej z~metod w~bibliotece \emph{fit-lib}. Zaprezentowana zawartość jest dokładnie taka sama, jak w~przypadku listingu \ref{lst:code_documentation}, natomiast przedstawiona w~bardziej przystępny sposób. Dokumentacja wygenerowana za pomocą narzędzia Doxygen świetnie nadaje się na udostępnienie zewnętrznym użytkownikom. Pozwala również w~łatwiejszy sposób przeglądać pełną dokumentację danego modułu bez potrzeby przeglądania kodu źródłowego.

\begin{figure}[H]
    \centering
    \includegraphics[width=\textwidth]{doxygen.png}
    \caption{Przykładowa dokumentacja metody w~bibliotece \emph{fit-lib} w~ramach projektu GGSS}
    \label{fig:doxygen}
\end{figure}

Ostatnim elementem dokumentacji zawartym w~projekcie są dokumenty \emph{how-to}. Napisane, podobnie jak pliki \emph{README}, za pomocą języka znaczników \emph{Markdown}, natomiast mają charakter globalny dla całego projektu - nie ograniczają się do jednego repozytorium. Dokumenty takie znajdują się w~repozytorium \emph{ggss-aux}. Opisane są tam krok po kroku bardziej zaawansowane aspekty pracy z~projektem GGSS, jak np.: sposób obsługi architektury wielopoziomowej opartej o~submoduły, czy też przygotowywanie wirtualnej maszyny do pracy jako GitLab Runner w~środowisku GitLab udostępnionym w~ramach infrastruktury CERN.

\section{Konwencja kodowania}
Ze względu na to, że w~trakcie pracy magisterskiej bardzo duży nacisk położono na część aplikacyjną projektu autorzy, jeszcze przed rozpoczęciem pracy nad kodem źródłowym, postanowili ustanowić konwencję kodowania, tak, aby na przestrzeni całego projektu GGSS utrzymać jednolity kod. Zasady, które zostały ustalony tyczą się nazewnictwa: klas, przestrzeni nazw, zmiennych, plików. Postanowiono wykorzystać, dobrze znane w~środowisku, systemy notacji ciągów tekstowych \emph{lower camel case} oraz \emph{upper camel case}. Ze względu na różnorodność możliwych rozszerzeń plików w~przypadku języka C++ postanowiono również ujednolicić ten aspekt. W~przypadku plików z~kodem źródłowym zastosowano rozszerzenia \emph{.cpp} oraz \emph{.h}. Ustanawiając konwencję kodowania postanowiono ograniczyć się do wyżej wymienionych aspektów, sposób projektowania architektury, podziału na foldery, klasy, etc. wewnątrz danego modułu pozostawiono bez większych obostrzeń. Oczywiście autorzy w~każdym z~dotkniętych miejsc stosowali dobre praktyki programistyczne oraz tak zwany \emph{clean code}, natomiast, ze względu na to, że w~większości przypadków prace nad projektem dotyczyły modyfikacji już istniejącego kodu oraz modułów była zachowana wcześniej zastosowana architektura.


\chapter{Prace nad architekturą i infrastrukturą projektu}
\label{cha:infra}


Niniejszy rozdział zawiera opis prac wykonanych przez autorów w ramach rozwoju architektury i infrastruktury systemu GGSS. Rozdział ten stanowi bezpośrednią kontynuację pracy inżynierskiej autorów, gdzie przygotowane zostały pierwsze wersje rozwijanych w ramach pracy magisterskiej rozwiązań. Przedstawione tu informację dotyczą szerokiego zakresu zagadnień związanych z inżynierią oprogramowania, takich jak: zarządzanie strukturą projektu oraz jego zależnościami, automatyzacja procesów towarzyszących wytwarzaniu oprogramowania czy przygotowanie infrastruktury ułatwiającej testy warstwy sprzętowej systemu. 

\section{Zmiany w architekturze projektu}
Przez zmiany w architekturze projektu autorzy rozumieją stopniowy rozwój rozwiązania przygotowanego w ramach napisanej przez nich pracy inżynierskiej. Wprowadzone po jej zakończeniu modyfikacje to przede wszystkim uproszczenie powstałej hierarchii zależności między poszczególnymi elementami warstwy oprogramowania (rozumianymi zarówno jako repozytoria, jak i biblioteki), uczynienie systemu bardziej przystępnym dla użytkownika (np. poprzez nadanie komponentom nazw dobrze oddających ich przeznaczenie) oraz przygotowanie systemu pozwalającego w prosty sposób odtworzyć kod źródłowy w wersji bez wprowadzonych w ramach pracy magisterskiej modyfikacji (jako rodzaj zabezpieczenia przed skutkami potencjalnych błędów, które mogły zostać wprowadzone do oprogramowania podczas prac nad nim). Znaczna część zmian opisanych w niniejszej części pracy była możliwa do wprowadzenia z uwagi na trwające jednocześnie prace nad kodem źródłowym systemu GGSS i zmiany zachodzące w ich czasie. 


\subsection{Początkowa architektura projektu}
Przeprowadzone przez autorów w ramach pracy inżynierskiej modyfikacje architektury systemu GGSS obejmowały przede wszystkim migrację projektu do systemu kontroli wersji Git, wprowadzenie spójnego nazewnictwa poszczególnych komponentów oraz zastosowanie funkcjonalności submodułów będącej częścią technologii Git do stworzenia hierarchicznej struktury repozytoriów (w odróżnieniu od pierwotnej, płaskiej architektury opartej o katalogi). Celem tych zmian było ułatwienie pracy nad pojedynczymi komponentami projektu oraz uczynienie struktury projektu przyjazną dla użytkownika, co zostało zdaniem autorów osiągnięte. 

Architektura stanowiąca punkt wyjściowy zmian wykonanych w ramach niniejszej pracy przedstawiona została na rysunku \ref{fig:old_structure} (z pominięciem repozytoriów pomocniczych, zawierających np. dokumentację). Projekt składał się więc pierwotnie z 14 repozytoriów zawierających wchodzące w skład oprogramowania systemu GGSS biblioteki, aplikacje i skrypty. Przygotowane w ten sposób rozwiązanie charakteryzowało się jednak pewnymi wadami i ograniczeniami, z których najważniejsze to:
\begin{itemize}
    \item głęboka hierarchia zależności, mająca negatywny wpływ na wydajność działania mechanizmu submodułów
    \item istnienie repozytorium \emph{ggss-misc}, zawierającego (poza szablonami CMake) elementy kodu źródłowego niepasujące do pozostałych bibliotek wchodzących w skład systemu: bazowe klasy wyjątków stosowanych w całym projekcie oraz flagi konfigurujące projekt w zależności od systemu operacyjnego (konieczność zastosowania tego typu zabiegu wynikła wprost z założenia o niemodyfikowaniu kodu źródłowego w czasie tworzenia pracy inżynierskiej)
    \item zachowanie oryginalnych nazw bibliotek i aplikacji, dostosowując je jedynie do przyjętej konwencji. Jedną z bibliotek wchodzących w skład projektu była biblioteka statyczna \emph{handle-lib}, odpowiedzialna za implementację mechanizmu slotów i sygnałów (\textbf{pisownia}), na co, zdaniem autorów, jej nazwa nie wskazuje.
    \item wnioskowanie o zależnościach pomiędzy bibliotekami na podstawie dyrektyw preprocesora \emph{include} zawartych w kodzie źródłowym, a nie wykorzystywanych funkcjonalności, co wynikało z niewielkiego doświadczenia i wiedzy autorów na temat systemu podczas tworzenia pracy inżynierskiej oraz wspomnianego już założenia o niemodyfikowaniu kodu źródłowego.
    \item założenie o tworzeniu oddzielnego repozytorium dla każdej z występujących w projekcie aplikacji, niezależnie od jej rozmiarów, co ostatecznie znacznie skomplikowało powiązania pomiędzy repozytoriami (np. repozytoria \emph{external-caen-n957-demo} oraz \emph{mca-n957} charakteryzują się podobnymi zależnościami i oba zawierają niewielkie aplikacje, których zadaniem jest współpraca z wielokanałowym analizatorem amplitudy CAEN N957 - mogłoby być więc połączone w jedno repozytorium).
    \item brak łatwego sposobu na odtworzenie pierwotnej postaci kodu źródłowego - mechanizm ten nie był potrzebny na etapie pracy inżynierskiej, ponieważ nie dokonywano wtedy modyfikacji we wspomnianym kodzie.
\end{itemize}

\begin{landscape}

\begin{figure}[H]
\centering
\includegraphics[width=1.4\textwidth]{components/infra_images/old_structure.pdf}
\caption{}
\label{fig:old_structure}
\end{figure}

\end{landscape}

\subsection{Uproszczenie architektury projektu}
Pierwszym podjętym przez autorów działaniem mającym na celu modyfikację struktury projektu była próba jej uproszczenia poprzez analizę zależności wewnętrzych systemu (tzn. zależności pomiędzy poszczególnymi bibliotekami). Należy tutaj zwrócić uwagę, że prac tych nie wykonywano przez pierwsze pół roku od rozpoczęcia przez autorów studiów magisterskich - skupienie się na użytkowaniu stworzonej architektury pozwoliło autorom zarówno na zapoznanie się lepiej z projektem, jak również na samodzielną obserwację jej zalet i wad. 

W celu zrozumienia wprowadzonych w projekcie zmian, konieczne jest zrozumienie toku rozumowania, jakim posługiwali się autorzy pracy podczas tworzenia oryginalnej struktury projektu - dotyczy to przede wszystkim repozytoriów \emph{ggss-lib}, \emph{ggss-software-libs}, \emph{ggss-hardware-libs}, \emph{ggss-util-libs} oraz \emph{ggss-misc}. Ich rola w projekcie prezentuje się następująco:
\begin{itemize}
    \item \emph{ggss-hardware-libs} - przechowywanie bibliotek odpowiedzialnych za obsługę urządzeń wchodzących w skład warstwy sprzętowej systemu GGSS. W pierwotnej wersji projektu były to następujące biblioteki:
    \begin{itemize}
        \item \emph{caenhv-lib} oraz \emph{caenn1470-lib} - odpowiedzialne za komunikację z zasilaczami wysokiego napięcia CAEN N1470
        \item \emph{mca-lib} oraz \emph{ortecmcb-lib} - odpowiedzialne za obsługę wielokanałowego analizatora amplitudy CAEN N957
        \item \emph{usbrm-lib} - odpowiedzialna za obsługę multipleksera sygnałów analogowych
    \end{itemize}
    \item \emph{ggss-software-libs} - przechowywanie bibliotek odpowiedzialnych za implementację wykorzystywanych przez system algorytmów i struktur danych związanych ściśle z warstwą oprogramowania (tzn. nie mających związku z warstwą sprzętową). W pierwotnej wersji projektu były to następujące biblioteki:
    \begin{itemize}
        \item \emph{xml-lib}
        \item \emph{fifo-lib}
        \item \emph{fit-lib}
        \item \emph{daemon-lib}
    \end{itemize}
    \item \emph{ggss-util-libs} - przechowywanie bibliotek, od których zależne są zarówno komponenty odpowiedzialne za obsługę warstwy sprzętowej projektu, jak i związane wyłącznie z warstwą oprogramowania. Innymi słowy, są to biblioteki nie mogące znaleźć się w żadnym z dwóch wymienionych powyżej repozytoriów. W pierwotnej wersji projektu były to:
    \begin{itemize}
        \item \emph{log-lib}
        \item \emph{utils-lib}
        \item \emph{handle-lib}
        \item \emph{thread-lib}
    \end{itemize}
    \item \emph{ggss-misc} - 
    \item \emph{ggss-lib} - przechowywanie kodu źródłowego zawierającego główną logikę systemu GGSS
\end{itemize}

Prace nad kodem źródłowym projektu pozwoliły autorom zaobserwować, iż pewna część występujących w nim dyrektyw preprocesora \emph{include} nie oddaje w poprawny sposób faktycznej struktury zależności między bibliotekami. Najważniejszy przykład stanowi łańcuch zależności występujących pomiędzy biblioteką \emph{ggss-lib}, a bibliotekami \emph{caenhv-lib} oraz \emph{thread-lib}. W oryginalnej wersji projektu zależności między wymienionymi komponentami prezentowały się tak, jak na rysunku \ref{fig:dependency_problem_old}, tzn. bibliteka \emph{ggss-lib} zależna była od biblioteki \emph{caenhv-lib}, która natomiast zawierała dyrektywę \emph{include} dołączającą plik nagłówkowy z biblioteki \emph{thread-lib}. 


\savebox{\mybox}{\includegraphics[width=0.40\textwidth]{components/infra_images/dependency_problem_old.pdf}}

\begin{figure}[H]
\centering
\begin{subfigure}[t]{0.40\textwidth}
\centering
\usebox{\mybox}
\caption{Test}
\label{fig:dependency_problem_old}
\end{subfigure}
\hfill
\begin{subfigure}[t]{0.55\textwidth}
\centering
\vbox to \ht\mybox{%
    \vfill
    \includegraphics[width=\textwidth]{components/infra_images/dependency_problem_solution.pdf}
    \vfill
}
\caption{Test}
\label{fig:dependency_problem_solved}
\end{subfigure}

\caption{Test}
\end{figure}

W rzeczywistości biblioteka \emph{caenhv-lib} nie wykorzystywała zawartości wspomnianego pliku nagłówkowego - pełniła jedynie formę swego rodzaju pośrednika, udostępniając znajdujące się tam klasy bibliotece \emph{ggss-lib}. Przeniesienie dyrektywy \emph{include} do biblioteki \emph{ggss-lib} spowodowało, iż żadna z bibliotek wchodzących w skład repozytorium \emph{ggss-hardware-libs} nie zawierała zależności do biblioteki \emph{thread-lib}. Rozwiązanie to pozwoliło dokonać migracji tejże biblioteki, wraz z wykorzystywaną przez nią biblioteką \emph{handle-lib}, do repozytorium \emph{ggss-software-libs}, redukując tym samym liczbę bibliotek znajdujących się w repozytorium \emph{ggss-util-libs}. Rysunek \ref{fig:dependency_problem_solved} przedstawia w sposób schematyczny strukturę otrzymanego rozwiązania.

W związku z opisanymi powyżej zmianami ilość kodu źródłowego znajdującego się w repozytorium \emph{ggss-util-libs} znacznie spadła - pozostałe tam biblioteki \emph{log-lib} oraz \emph{utils-lib} charakteryzowały się niewielkim rozmiarem. Spowodowało to, iż jednoczesne istnienie modułów \emph{ggss-misc} oraz \emph{ggss-util-libs} (po wprowadzonych zmianach spełniających tą samą rolę przechowywania niewielkiej liczby komponentów wykorzystywanych przez wiele modułów projektu GGSS) przestało być uzasadnione. Kolejny etap wykonanych prac stanowiło więc przeprowadzenie integracji tychże repozytoriów - w tym celu zdecydowano się na likwidację modułu \emph{ggss-misc} po wcześniejszym przeniesieniu jego zawartości do \emph{ggss-util-libs}.

Migracja znajdujących się w repozytorium \emph{ggss-misc} plików \emph{.cmake} (modułów wykorzystywanych przez infrastrukturę budowania projektu) wymagała, poza wykonaniem trywialnej czynności przeniesienia katalogu, aktualizacji (na poziomie całego projektu) ścieżek wskazujących lokalizację tychże plików. Działanie to było konieczne, ponieważ narzędzie CMake wymaga od programisty, by wyspecyfikował on lokalizację modułów \emph{.cmake} dołączanych do projektu (np. za pomocą komendy \lstinline{include()}) poprzez dodanie ścieżki z ich lokalizacją do listy \lstinline{CMAKE_MODULE_PATH} (przykład wykorzystania tejże listy przedstawiony został na listingu \textbf{listing}). Oznaczało to więc konieczność wykonania, w każdym module wykorzystującym pliki \emph{.cmake}, zmiany wspomnianej ścieżki tak, by wskazywała na katalog \emph{cmake-templates} w repozytorium \emph{ggss-util-libs}.

Poza wspomnianymi plikami \emph{.cmake} w repozytorium \emph{ggss-misc} znajdował się katalog \emph{include}, zawierający trzy pliki nagłówkowe z kodem napisanym w języku C++:
\begin{itemize}
    \item pliki \lstinline{ggssExceptions.h} oraz \lstinline{HardwareException.h} zawierające klasy bazowe wyjątków wykorzystywanych w całym projekcie GGSS
    \item plik \lstinline{CompatibilityFlags.h}, zawierający flagi konfigurujące projekt w zależności od platformy docelowej (Windows lub Linux)
\end{itemize}
Pliki te nie wchodziły oryginalnie w skład żadnej z bibliotek projektu GGSS, nie mogły zostać do nich również dodane przez autorów podczas przygotowywania pracy inżynierskiej, ponieważ wymagałoby to modyfikacji kodu źródłowego systemu. Podczas przeprowadzanej w ramach niniejszej pracy migracji tych plików do repozytorium \emph{ggss-util-libs} zdecydowano się na likwidację katalogu \emph{include} i rozdysponowanie jego zawartości do istniejących lub nowych bibliotek. Plik \lstinline{CompatibilityFlags.h} przeniesiony został więc do biblioteki \emph{utils-lib}, natomiast na potrzebę dwóch pozostałych nagłówków przygotowana została nowa biblioteka \emph{exceptions-lib}. 

Finalna struktura repozytorium \emph{ggss-util-libs} przedstawiona została na listingu \textbf{listing}. Poza wspomnianymi do tej pory zmianami nowość stanowi katalog \lstinline{doxygen-config}, zawierający prosty plik konfigurujący działanie narzędzia Doxygen \textbf{cytat} służącego do generowania dokumentacji programów napisanych w języku C++. Rozszerzenie projektu o możliwość generowania dokumentacji zostanie jednak opisane szczegółowo w dalszej części pracy (sekcja \textbf{dodac referencje do sekcji}).


Poza wspomnianymi do tej pory repozytoriami zmianami objęte zostały ponadto moduły przechowujące aplikacje służące do testowania i obsługi urządzeń elektronicznych wchodzących w skład warstwy sprzętowej systemu GGSS. Motywacją do wprowadzenia modyfikacji była konieczność rozbudowy projektu o kolejne tego typu aplikacje (co zostanie szerzej opisane w sekcji \textbf{sekcja}) - tworzenie dla każdej z nich osobnego repozytorium znacząco komplikowałoby strukturę projektu. Zdecydowano zatem, iż repozytoria \emph{mca-n957} oraz \emph{external-caen-n957-demo} zostaną dołączone do nowo powstałego repozytorium \emph{ggss-hardware-service-apps}, grupującego niewielkie programy służące do operowania na urządzeniach.


% TODO: moze wspomniec o koniecznosci zachowania historii i jak zostało to zrobione???

Poza zmniejszeniem progu wejścia do projektu poprzez uczynienie jego struktury prostszą, opisane do tej pory zmiany korzystnie wpłynęły na działanie mechanizmu submodułów systemu Git, na którym oparty został proces zarządzania zależnościami między repozytoriami w projekcie. Redukcja liczby repozytoriów i powiązań między nimi oraz zmniejszenie głębokości drzewa zależności (poprzez likwidację repozytorium \emph{ggss-misc}) miało pozytywny wpływ na wydajność systemu zarządzającego architekturą projektu. 



\subsection{Dodanie możliwości odtworzenia pierwotnej wersji kodu źródłowego}
% branche legacy, dokumentacja

\subsection{Pomniejsze zmiany}
Poza do tej pory opisanymi, wykonanych zostało kilka pomniejszych modyfikacji mających na celu szeroko pojętą poprawę jakości struktury projektu. Przeprowadzone prace obejmują bogaty zakres wprowadzonych zmian, nie jest więc możliwe zamieszczenie w niniejszej pracy dokładnego opisu każdej z nich. Poniżej krotko opisane zostały więc trzy wybrane przez autorów modyfikacje, charakteryzujące się różnym poziomem skomplikowania, ale operujące na poziomie pojedynczych repozytoriów. 

\subsubsection{Likwidacja repozytorium \emph{ggss-oper}}
Jednym z repozytoriów wprowadzonych przez autorów w ramach wykonywania pracy inżynierskiej był moduł \emph{ggss-oper}, zawierający skrypty oraz pliki konfiguracyjne stanowiące znaczną część infrastruktury przeznaczonej do użytkowania wraz z oprogramowaniem GGSS na maszynie docelowej. Zawartość tego repozytorium, nie stanowiąca wkładu wniesionego przez autorów niniejszej pracy w system, obejmowała m.in.: 
\begin{itemize}
    \item pierwsze wersje skryptów służących do przeprowadzania testów urządzeń wchodzących w skład warstwy sprzętowej projektu (napisane z wykorzystaniem języka Python)
    \item skrypty zarządzające stanem środowiska docelowego (np. ustawiające wymagane zmienne środowiskowe)
    \item skrypty zarządzające oprogramowaniem systemu GGSS, np. \emph{ggss\_monitor.sh} pozwalający na uruchamianie, zatrzymywanie oraz sprawdzanie stanu aplikacji \emph{ggss-runner}
\end{itemize}
Wraz z postępami prac nad projektem, część z wymienionej powyżej zawartości zastąpiona została przez autorów pracy rozwiązaniami alternatywnymi (np. skrypty służące do przeprowadzania operacji na urządzeniach zastąpione zostały aplikacjami napisanymi w języku C++), pozostałe przeniesione zostały natomiast do repozytorium \emph{ggss-all}. Ostatecznie moduł został więc zlikwidowany.


\subsubsection{Utworzenie biblioteki \emph{asyncserial-lib}}
Podczas prac nad kodem źródłowym bibliotek statycznych wchodzących w skład repozytorium \emph{ggss-hardware-libs} zaobserwowano, że w katalogach bibliotek \emph{usbrm-lib} oraz \emph{caenn1470-lib} zamieszczony został, poza właściwym dla nich kodem źródłowym, zestaw plików zawierających implementację asynchronicznej komunikacji z urządzeniami za pomocą interfejsu szeregowego. Ponieważ znalezione w obu przypadkach pliki nie różniły się od siebie, i jednocześnie stanowiły niezbędny element wspomnianych komponentów systemu (zawierały kluczową dla działania projektu funkcjonalność), zdecydowano o utworzeniu nowej biblioteki zawierającej omawiane pliki. Biblioteka nazwana została, zgodnie ze swoim przeznaczeniem, \emph{asyncserial-lib} i weszła w skład repozytorium \emph{ggss-hardware-libs}.


\subsubsection{Zmiana nazwy biblioteki \emph{handle-lib}}
Jedną z bibliotek będących częścią systemu GGSS była biblioteka \emph{handle-lib}, odpowiedzialna za implementację mechanizmu slotów i sygnałów \textbf{napisac moze co to}. Oryginalnie biblioteka ta znajdowała się w repozytorium \emph{ggss-util-libs}, jednak wraz z postępem prac przeniesiona została, wraz z biblioteką \emph{thread-lib}, do repozytorium \emph{ggss-software-libs} (powód tej migracji opisany został w sekcji \textbf{podaj sekcje} niniejszej pracy). Nazwa biblioteki nie pozwalała użytkowniki domyślić się, jakie jest jej zastosowanie - z tego powodu zdecydowano się wprowadzić nową nazwę: \emph{sigslot-lib} (od angielskiego \emph{signals and slots}).


\subsection{Podsumowanie: ostateczna struktura projektu}





\clearpage
\section{Automatyzacja pracy z submodułami (JC)}
\label{sec:gitio}

Ninejszy rozdział jest poświęcony obsługi wielopoziomowej struktury opartej o \emph{git submodules} obecnej w projekcie GGSS. Przedstawione zostaną plusy oraz minusy zastosowanego w trakcie pracy inżynierskiej rozwiązania. Omówiona zostanie przygotowana przez autorów infrastruktura mająca na celu ułatwienie pracy z submodułami. Dodatkowo krótko zostanie opisane przygotowane \emph{how-to} oraz praktyki które powinno się stosować pracując z taką architekturą.

\subsection{Wprowadzenie do problematyki}

W trakcie pracy inżynierskiej, a konkretnie migracji całego projektu GGSS do systemu kontroli wersji \emph{git} zdecydowano się na wykorzystanie technologii \emph{git submodules}. Ze względu na nacisk na zwiększenie modularyzacji projektu technologia ta idealnie wpasowywała się w docelową architekurę. Zasada działania submodułów jest bardzo zbliżona do dowiązań symbolicznych stosowanych między innymi w systemach UNIX. Zamiast wskazywać na ścieżkę do folderu na lokalnym systemie submoduł wskazuje na ścieżkę do konkretnej wersji repozytorium na zewnętrznym serwerze od którego zależy nasz moduł. Rysunek \ref{fig:submodules_links} przedstawia zasadę działania submodułów oraz wpływ wersjonowania na tenże mechanizm. Wykorzystanie submodułów pozwala na w pełni odseparowaną pracę nad wybranym komponentem systemu. Nie potrzebujemy pobierać żadnych dodatkowych plików, czy też zależności w celu zmienienia kodu źródłowego. Rozwiązanie to pozwala też na skorzystanie z bardzo szybkiej inicjalizacji całego projektu jedną komendą, co zostało przedstawione w listingu \ref{lst:initialize}.

\begin{figure}[H]
    \centering
    \includegraphics[width=0.9\textwidth]{submodule_links.pdf}
    \caption{Zasada działania submodułów.}
    \label{fig:submodules_links}
\end{figure}

\begin{lstlisting}[language=c++, caption={Inicjalizacja pełnej sturktury projektu jedną komendą.}, label={lst:initialize}]
root@host:/# git clone ssh://git@gitlab.cern.ch:7999/atlas-trt-dcs-ggss/ggss-all.git && cd ggss-all && git submodule update --init --recursive
Cloning into '/CERN/ggss-all/ggss-dim-cs'...
Cloning into '/CERN/ggss-all/ggss-driver'...
Cloning into '/CERN/ggss-all/ggss-oper'...
Cloning into '/CERN/ggss-all/ggss-runner'...
Cloning into '/CERN/ggss-all/ggss-spector'...
Cloning into '/CERN/ggss-all/mca-n957'...
Cloning into '/CERN/ggss-all/ggss-dim-cs/external-dim-lib'...
Cloning into '/CERN/ggss-all/ggss-dim-cs/ggss-misc'...
Cloning into '/CERN/ggss-all/ggss-driver/external-n957-lib'...
Cloning into '/CERN/ggss-all/ggss-driver/ggss-misc'...
...(13 lines truncated)
\end{lstlisting}


\subsection{Motywacja do wprowadzenia zmian}

Pomimo wielu aspektów \emph{git submodules}, które bardzo dobrze wpasowały się w, kreowaną przez autorów w trakcie pracy inżynierskiej, sturkturę technologia ta posiada też swoje minusy. Pierwszy znaczącym problemem napotkanym w trakcie pracy z submodułami było nietypowe zachowanie repozytoriów w trakcie ich inicjalizacji, a konkretnie automatyczne odłączanie ich od głównej gałęzi. Co więcej praca z submodułami wymaga od programisty zwiększonej czujności oraz stosowania dodatkowych zasad, ponieważ więcej jest miejsc na pomyłkę, co może doprowadzić do niepoprawnego działania wykorzystanych narzędzi. Kolejnym problemem napotkanym w trakcie pracy z submodułami jest czasochłonność niektórych operacji, w szczególności aktualizacji repozytorium na samym dole ``drzewa zależności``. Zmiana taka wymaga ręcznej aktualizacji po kolei każdego z repozytorium, aż do samej góry tejże skruktury co przedstawia rysunek \ref {fig:submodules_update}. Każda z aktualizacji przedstawiona na wyżej wymieionym rysunku, to tak na prawdę cztery lub więcej akcji do których wliczają się: aktualizacja repozytorium podrzędnego, dodanie wszystkich zmian do rejestru odpowiedzialnego za ich śledzenie, utworzenie nowej wersji repozytorium, opublikowanie nowej wersji na zewnętrznym serwerze.



\clearpage
\begin{figure}[H]
    \centering
    \includegraphics[width=0.9\textwidth]{submodules_update.pdf}
    \caption{Przykładowa architektura oparta o submoduły z krokami jakie należy podjąć, aby wprowadzić zmiany na ``najniższym`` poziomie.}
    \label{fig:submodules_update}
\end{figure}


\subsection{Automatyzacja z użyciem GITIO}

Problemem, którego rozwiązanie pochłonęło najwięcej czasu i wymagało największego wkładu pracy przez autorów było monotonne, wielokrokowie wprowadzanie zmian do projektu, szczególnie u dołu struktury zależności. W celu rozwiązanie tego problemu przygotowano aplikacje \emph{gitio} z wykorzystaniem języka \emph{Python}. Ze względu na to, że metadane technologii \emph{git} są bardzo złożone, a opanowanie zasad węwnętrznego działania tejże technologii wymagałoby bardzo dużo czasu skorzystano z dedykowanej, do tej technologii, bilbioteki napisanej również w języku \emph{Python}.

Zasada działania aplikacji jest dosyć prosta, natomiast znacząco ułatwia działania z wielopoziomową strukturą opartą o \emph{git submodules}. Argumenty wejściowe jako przyjmuje \emph{gitio} to:
\begin{itemize}
    \item \lstinline{-h, --help} - pozwala na wyświetlnie informacji o przeznaczeniu programu oraz przyjmowanych argumentach wraz z krótkim opisem
    \item \lstinline{-p PATH, --path PATH} -
    \item \lstinline{-b BIN, --bin BIN} -
\end{itemize}

\subsection{Dokumentacja sposobu pracy z submodułami}



\section{Rozwój systemu budowania projektu}
\section{Automatyzacja i centralizacja wersjonowania projektu}
\section{Pakietowanie i rozlokowanie projektu}
\section{Rozwój infrastruktury do testowania warstwy sprzętowej}


\chapter{Prace nad kodem źródłowym projektu}
\label{cha:code}

% - wprowadzenie do projektu (technologie, ograniczenia itp.)
% - metodyka prac
% - poprawa jakości kodu źródłowego
% - wprowadzenie nowych funkcjonalności
\chapter{Testy systemu (AK i JC)}
\label{cha:tests}

\section{Cykliczne testy systemu (AK)}
% Specyfika (co ile, w jakich przypadkach byly wykonywane)
% Wspomniec ze opisane w inz, testowane rozne wersje, ze dzialaja, ktora uzywana najczesciej
% Co bylo testowane (poprawnosc dzialania przez kilka godzin/kilka dni, pojedyncze funkcjonalnosci w zaleznosci od potrzeb, zuzycie pamieci za pomoca skryptow + listing)
% Opisac krotko procedure (moze jakis diagram)

\section{Testy po migracji systemu (JC)}
% Opisac krotko co to za migracja
% Jak miala wygladac, a jak wygladala
% Wklad w migracje
% Testy hardware + na co sie przydaly
% Uruchomienie glownej aplikacji ggss

\section{Testy wersji finalnej (AK i JC)}
% Wstep, kiedy wykonywane, w jakim czasie (AK)
% Kazda funkcjonalnosc z osobna
%   - po kolei opis kazdej (np. hvkiller, komendy itp)
% Testy hardware (skrypty)
% Testy zuzycia zasobow
\chapter{Podsumowanie (AK i JC)}
\label{cha:summary}

% zastosowano praktyki zblizone do tych stosowanych w duzych projektach informatycznych
% ale zeskalowane - wykorzystanie doswiadczenia autorow

% niektore z wprowadzonych rozwiazan nie sa co prawda optymalne, bo srodowisko, ale dzialaja

% pierwsze zderzenie z kodem zrodlowym
% wszystkie funkcjonalnosci zaimplementowane

% zastosowane zostaly srodki majace zapewniec niezawodnosc
% powyzsze potwiedza fakt dzialania systemu (pomyslnie zakonczone testy)
% co sie sprawdzilo (testy jednostkowe, automatyzacja CI/CD, aspekty organizacyjne - opisane szerzej w dodatku)

% projekt pozostawiony wraz z szeregiem instrukcji ulatwiajacych promotorowi wdrozenie



\printbibliography

\end{document}