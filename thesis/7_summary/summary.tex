\chapter{Podsumowanie (AK i JC)}
\label{cha:summary}

W niniejszym manuskrypcie zaprezentowane zostały wykonane przez autorów prace nad rozbudową i uaktualnieniem warstwy oprogramowania Systemu Stabilizacji Wzmocnienia Gazowego detektora ATLAS TRT. Zaprezentowane w kolejnych rozdziałach zmiany obejmowały szereg zróżnicowanych zagadnień, związanych zarówno z infrastrukturą projektu, jak i jego kodem źródłowym. 

Podczas wykonywania zaprezentowanych zadań autorzy zapoznali się z kodem źródłowym projektu. W pierwszych etapach prac wykonywano przede wszystkim niewielkie poprawki, których celem było poznanie systemu. Następnie autorzy rozszerzyli projekt o nowe funkcjonalności. Wszystkie założenia dotyczące wprowadzanych do systemu modyfikacji zostały spełnione, a każda wymagana funkcjonalność została zaimplementowana. Z punktu widzenia współczesnych praktyk programistycznych nie wszystkie zaproponowane rozwiązania są optymalne, jednakże wynika to przede wszystkim z nakładanych przez środowisko, w jakim działa projekt, ograniczeń.

W celu skutecznej organizacji projektu autorzy wykorzystali swoje doświadczenie pozyskane podczas prac nad komercyjnymi projektami informatycznymi. Stąd też duży nacisk położony został na odpowiednią organizację pracy oraz zastosowanie znanych praktyk ułatwiających pracę nad złożonymi systemami informatycznymi. Autorzy dokonali w nich oczywiście odpowiednich modyfikacji, by dostosować je do mniejszej skali projektu.

Podczas prac nad systemem autorzy starali się dołożyć wszelkich starań, by zapewnić jego niezawodność. O skuteczności tych działań świadczą przeprowadzone przez nich testy, jednoznacznie wskazujące na poprawność działania projektu. Szczególnie pomocne w zachowaniu odpowiedniej jakości okazały się regularne testy w środowisku docelowym, stosowanie testów automatycznych oraz rozbudowana infrastruktura automatyzująca czynności związane z cyklem życia oprogramowania.

Projekt pozostawiony został przez autorów w stanie umożliwiającym jego łatwy rozwój i utrzymanie. Aby dodatkowo ułatwić to zadanie, przygotowany został szereg instrukcji i poradników opisujących, w jaki sposób wykonywać zarówno podstawowe, jak i bardziej złożone działania. Praca nad systemem GGSS pozwoliła autorom na zdobycie cennego doświadczenia, zarówno z uwagi na prowadzoną pracę zespołową, jak i międzynarodowy charakter projektu. 
