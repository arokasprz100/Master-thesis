\chapter{Wstęp (AK i~JC)}
\label{cha:wstep}

\section{Wprowadzenie do domeny projektu (AK)}

Europejska Organizacja Badań Jądrowych CERN jest jednym z~najważniejszych i~najbardziej znanych ośrodków naukowo-badawczych na świecie oraz miejscem intensywnego rozwoju fizyki i~informatyki. W~CERN-ie powstało wiele znaczących technologii, w~tym między innymi stanowiący podstawę sieci WWW (\emph{World Wide Web}) \cite{WWW_wikipedia} protokół HTTP (\emph{Hypertext Transfer Protocol}) \cite{HTTP_wikipedia}. Ośrodek ten kojarzony jest dziś jednak przede wszystkim z~największym akceleratorem cząstek na świecie - Wielkim Zderzaczem Hadronów (LHC - \emph{Large Hadron Collider}) \cite{LHC_cern}, oraz z~działającymi przy nim eksperymentami. Jednym z~tych eksperymentów jest pełniący kluczową rolę w~rozwoju współczesnej fizyki detektor ATLAS (\emph{A Toroidal LHC ApparatuS}) \cite{ATLAS_cern}.

Detektor ATLAS zbudowany jest z~kilku pod-detektorów, tworzących strukturę warstową. Najbardziej wewnętrzną część stanowi tzw. Detektor Wewnętrzny (ang. \emph{Inner Detector}) \cite{ATLAS_agh}, składający się z~kolei z~trzech kolejnych podsystemów. Jednym z~tychże podsystemów, szczególnie istotnym w~kontekście niniejszej pracy, jest detektor promieniowania przejścia (TRT - \emph{Transition Radiation Tracker}).

System Stabilizacji Wzmocnienia Gazowego (GGSS - \emph{Gas Gain Stabilization System}) \cite{GGSS_jinst} jest jednym z~podsystemów detektora TRT, mającym zapewnić jego poprawne działanie. Projekt ten zintegrowany jest z~systemem kontroli detektora ATLAS (DCS - \emph{Detector Control System}) \cite{ATLAS_agh}. W~skład systemu GGSS wchodzi zarówno warstwa oprogramowania, jak i~szereg urządzeń. Ze względu na jego rolę, jednym z~najważniejszych wymagań stawianych przed projektem jest wysoka niezawodność.

\section{Cel i~założenia pracy (JC)}
Celem niniejszego dokumentu jest omówienie i~podsumowanie półtorarocznych prac nad rozwojem i~usprawnieniem systemu GGSS. Opisane zmiany stanowią kontynuację działań rozpoczętych w~ramach stworzonej przez autorów pracy inżynierskiej o~tytule \emph{Rozbudowa i~uaktualnienie oprogramowania systemu GGSS detektora ATLAS TRT} \cite{GGSS_inz}. Praca inżynierska skupiała się na aspektach infrastruktury oraz architektury projektu, takich jak migracja do systemu kontroli wersji Git \cite{Git_main} i~przebudowa architektury na bardziej modularną oraz prostszą do zrozumienia. Celem wielu wprowadzonych wtedy przez autorów zmian było udoskonalenie procesu wytwarzania oraz wdrażania oprogramowania w~środowisko produkcyjne, co zostało osiągnięte m.in. poprzez wykorzystanie technologii CMake \cite{CMake_main} oraz GitLab CI/CD \cite{CI_main}. 

Główny nacisk pracy magisterskiej położony został natomiast na część aplikacyjną projektu - kod źródłowy odpowiedzialny za główną logikę został rozbudowany oraz udoskonalony. W~ramach zmian w~kodzie zostały dodane nowe funkcjonalności, jego nieużywane fragmenty zostały usunięte z~projektu, szeroko rozumiana jakość została zwiększona, a~jego poprawne działanie zostało zabezpieczone poprzez testy automatyczne. Działania podjęte w~celu zapewnienia, że projekt działał będzie w~sposób niezawodny, takie jak testy automatyczne, manualne oraz przygotowanie potrzebnej do ich przeprowadzenia infrastruktury, stanowią bardzo istotną część niniejszego manuskryptu i~zostały w~nim szczegółowo omówione. Ponadto, ze względu na ciągłą pracę z~systemem, a~co za tym idzie poznawanie jego środowiska docelowego oraz newralgicznych punktów, część niniejszej pracy poświęcona została udoskonaleniu powstałych w~ramach pracy inżynierskiej rozwiązań związanych z~infrastrukturą oraz architekturą projektu.

Jednym z~postawionych autorom celów było odpowiednie udokumentowanie projektu tak, aby ewentualne przyszłe zmiany można było wykonywać z~jak największą łatwością, a~wprowadzenie nowych osób w~projekt było jak najprostsze. Oprócz obszernego opisu zawartego w~ramach tego manuskryptu wymogiem było, aby przygotować krótkie, lecz treściwe pliki instruktażowe, opisowe oraz odpowiednio udokumentować kod źródłowy.

Ze względu na bardzo szeroki zakres tematów podejmowanych w~niniejszej pracy zdecydowano się na podział, który odchodzi od standardowego. W~celu ułatwienia korzystania z~manuskryptu wprowadzenie do opisywanego problemu oraz wykonane prace zostały zamieszczone w~jednym miejscu. Zatem zarówno nakreślenie problemu, stan początkowy oraz sposób jego rozwiązania następują zaraz po sobie. Schemat ten został powtórzony dla każdego poruszanego w~pracy zagadnienia. Autorzy chcą w~ten sposób ułatwić wykorzystanie tegoż dokumentu zarówno jako wprowadzenia do tematyki, jak również jako dokumentacji stanowiącej podstawę ewentualnego dalszego rozwoju projektu.
