\chapter{Wykorzystane technologie (AK i JC)}
\label{ch:technologies}

% setup
\graphicspath{{2_technologies/static/}}

% content
% Krotki opis o czym bedzie rozdzial


\section{Język C++ (AK)}

% Krótkie omówienie historii, zwiazku z C, dostepnych paradygmatow
% Standardy języka C++, wynikające z nich korzyści i problemy
% Najważniejsze elementy standardu C++11 (krotko, ze sa: sprytne wskazniki, std::function, lambdy, wielowatkowosc, semantyka przenoszenia, jak to wplynelo na spolecznosc)
% Przykład różnicy miedzy standardem C++03 a C++11
% Przykład rożnicy między C++11 a C++20
% Dodatkowe biblioteki C i C++: Boost, GSL, Qt + Qwt
% Podac przyklad z Boosta, pokazac jak zintegrowany zostal w standardzie

\section{Język Python (AK)}
% Krotkie omownienie pythona jako jezyka o szerokim zastosowaniu (od skryptow po aplikacje webowe) - napisac jak my uzywamy
% Glowne zalety Pythona na przykladzie - przejrzysta skladnia, niski prog wejscia, elastycznosc

\section{Narzędzia do analizy oprogramowania (AK)}
% Wstep teoretyczny - czym jest analiza kodu, jakie mamy rodzaje, co to jest instrumentacja
% Omowienie Valgrinda jako frameworka, wskazanie narzedzia Memcheck
% Krotkie omowienie dzialania Memcheck, m.in. rodzaje wyciekow pamieci, jakie są ich konsekwencje, wskazac te potencjalnie grozne dla aplikacji dzialajacych jako uslugi (stopniowe nagromadzenie sie zuzytej pamieci - przyklad)
% Krotko o innych narzedziach (np. Cachegrind), czemu tutaj mniej istotne
% Statyczna analiza kodu źródłowego - jakie możliwości daje CLion oraz clang-tidy
% Wskazac mozliwości zastosowania analizy w wiekszym srodowisku (np. SonarLint i SonarQube w wielu firmach), możliwość wlaczenia takiego rozwiazania w CI + dlaczego u nas tego nie ma

\section{System kontroli wersji Git (JC)} % submoduly, repozytoria, gałęzie %
\section{Portal GitLab (JC)}
%    \subsection{GitLab CI/CD}
%    \subsection{GitLab Runner} % albo dodac przypis w miejscu w ktorym jest o tym wspomniane, ze zostalo to opisane w pracy inzynierskiej

\section{Narzędzie CMake (AK)}
% Krotki opis co to jest, czym sie rozni od Make, zalety
% Opis dwoch wersji, jakie sa roznice miedzy nimi
% Prosty przyklad
% Co to jest CPack i CTest, do czego sluza


\section{Menadżer pakietów RPM (JC)}
