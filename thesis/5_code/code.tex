\chapter{Prace nad kodem źródłowym projektu (AK)}
\label{cha:code}

Niniejszy rozdział stanowi opis wprowadzonych przez autorów zmian w kodzie źródłowym aplikacji \emph{ggss-runner}, stanowiącej trzon warstwy oprogramowania systemu GGSS. Opis poszczególnych modyfikacji poprzedzony został krótkim wprowadzeniem, opisującym wysokopoziomowe dzialanie omawianego programu oraz wynikające z jego specyfiki ograniczenia i założenia. Omówienie wprowadzonych zmian podzielone zostało na dwie części. W pierwszej z nich przedstawione zostały modyfikacje nie mające wpływu na sposób działania aplikacji, ale poprawiające jakość jej kodu źródłowego, np. poprzez jego migrację do nowszego standardu języka C++. Druga część stanowi natomiast opis nowych funkcjonalności oraz rozszerzeń wprowadzonych przez autorów do projektu. 

\section{Analiza aplikacji \emph{ggss-runner}}

\section{Specyfika pracy}

    \subsection{Testy jednostkowe} % w tym problem mockowania w systemach legacy
    \subsection{Zakres wprowadzanych zmian} % napisac ze celem nie bylo przepisanie calej apki
    \subsection{Przyjęte ograniczenia} % brak nowych zaleznosci, zgodnosc ze srodowiskiem docelowym

\section{Poprawa jakości kodu źródłowego}

    \subsection{Migracja do standardu C++11}
    \subsection{Naprawa błędów w kodzie źródłowym}
    \subsection{Likwidacja nieużywanych fragmentów kodu źródłowego}
    \subsection{Zmiany w strukturze bibliotek}

\section{Rozszerzenie możliwości aplikacji}
    
    \subsection{Obsługa zaawansowanych komend dla zasilaczy wysokiego napięcia}
    \subsection{Rozbudowa biblioteki odpowiedzialnej za dopasowywanie krzywej}
    \subsection{Zmiany w sposobie aktualizacji parametrów i zebranych danych}
    \subsection{Zabezpieczenie przed przepełnieniem bufora urządzenia MCA}
    \subsection{Wprowadzenie dodatkowych komend sterujących i monitorujących}

\section{Podsumowanie}
