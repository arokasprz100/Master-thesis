\chapter{Prace nad kodem źródłowym projektu (AK)}
\label{cha:code}

Niniejszy rozdział stanowi opis wprowadzonych przez autorów zmian w kodzie źródłowym aplikacji \emph{ggss-runner}, stanowiącej trzon warstwy oprogramowania systemu GGSS. Opis poszczególnych modyfikacji poprzedzony został krótkim wprowadzeniem, opisującym wysokopoziomowe dzialanie omawianego programu oraz wynikające z jego specyfiki ograniczenia i założenia. Omówienie wprowadzonych zmian podzielone zostało na dwie części. W pierwszej z nich przedstawione zostały modyfikacje nie mające wpływu na sposób działania aplikacji, ale poprawiające jakość jej kodu źródłowego, np. poprzez jego migrację do nowszego standardu języka C++. Druga część stanowi natomiast opis nowych funkcjonalności oraz rozszerzeń wprowadzonych przez autorów do projektu. 

\section{Analiza aplikacji \emph{ggss-runner}}

\section{Specyfika pracy}

\subsection{Testy jednostkowe} % w tym problem mockowania w systemach legacy
\subsection{Zakres wprowadzanych zmian} % napisac ze celem nie bylo przepisanie calej apki
\subsection{Przyjęte ograniczenia} % brak nowych zaleznosci, zgodnosc ze srodowiskiem docelowym

\section{Poprawa jakości kodu źródłowego}
% celem tego etapu bylo zapoznanie sie ze struktura kodu oraz jednoczesne zwiekszenie jego jakosci
% tego typu operacje wazne, bo pozwalaja innym latwiej zrozumiec kod
% napisac ze zmiany mialy charakter malych modyfikacji (nie calych architektur)

\subsection{Migracja do standardu C++11}
% wymienic krotko (np. od punktow) najwazniejsze z wprowadzanych zmian
% jakies 2-3 mniejsze przyklady (petla for, noexcept, enum class w log-lib)
% wspomniec, ze nie wszystko dalo sie zmienic (np. bo roznica miedzy std a boost, np watki)

\subsection{Naprawa błędów w kodzie źródłowym}
% wspomniec, ze bledow bylo bardzo niewiele, zaden z nich nie zagrazal tak naprawde poprawnosci dzialania projektu
% jako przyklad opisac blad z wyszukiwaniem w XML-u

\subsection{Likwidacja nieużywanych fragmentów kodu źródłowego}
% fragmenty takie jak w fit-lib (3 znalezione funkcje) i xml-lib (dwie implementacje przechowywania wezlow) - pozostałosci po starszych wersjach systemu i eksperymentach
% pozostalosci po starej wersji systemu w bibliotece ggss-lib
% szczegolny przyklad: metody lamiace enkapsulacje w bibliotece fifo-lib

\subsection{Pozostałe zmiany i podsumowanie}
% zmiana w strukturze bibliotek (rozbicie na pliki - log i sigslot)
% wyodrebnienie powtarzajacych sie fragmentow kodu (np. w log-lib i xml-lib)
% wspomniec ze ujednolicona zostala konwencja nazewnictwa i dokumentacji
% podsumowanie - ile bibliotek poddanych zostalo refactoringowi, ktore nie i dlaczego

\section{Rozszerzenie możliwości aplikacji}
    
\subsection{Obsługa zaawansowanych komend dla zasilaczy wysokiego napięcia}
\subsection{Rozbudowa biblioteki odpowiedzialnej za dopasowywanie krzywej}
\subsection{Zmiany w sposobie aktualizacji parametrów i zebranych danych}
\subsection{Zabezpieczenie przed przepełnieniem bufora urządzenia MCA}
\subsection{Wprowadzenie dodatkowych komend sterujących i monitorujących}

\section{Podsumowanie}
