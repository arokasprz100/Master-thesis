\chapter{Testy systemu (AK i JC)}
\label{cha:tests}
Niniejszy rozdział stanowi szczegółowy raport z przeprowadzanych w środowisku docelowym testów systemu GGSS. Został on podzielony na trzy części, opisujące różne rodzaje przeprowadzanych przez autorów sprawdzeń. Pierwsza z nich przybliża informacje dotyczące przeprowadzanych w sposób cykliczny testów - nacisk położony został tutaj przede wszystkim na opis powtarzanej w każdej iteracji procedury pozwalającej zweryfikować poprawność działania systemu. Druga część stanowi opis sprawdzeń wykonywanych w czasie mającej miejsce w lipcu 2021 roku migracji systemu do nowego środowiska docelowego. Zamieszczone w niej informacje dotyczą wkładu autorów we wspomnianą migrację, obejmującego m.in. wykonanie testów warstwy sprzętowej systemu GGSS. Ostatnia część niniejszego rozdziału opisuje wykonane w sierpniu 2021 roku testy finalnej wersji projektu. W tym przypadku przedstawiony został szczegółowy raport, obejmujący weryfikację poprawności działania każdej wprowadzonej do systemu lub zmodyfikowanej funkcjonalności, badanie stabilności systemu ze względu na wykorzystywane zasoby oraz testy nowych elementów infrastruktury, takich jak skryptu zarządzające środowiskiem docelowym.

\section{Cykliczne testy systemu (AK)}
Praca nad projektem stanowiącym część dużego, rozwijanego przez wiele osób systemu, wymaga stosowania metod pozwalających na zapewnienie jego niezawodności. Dlatego też autorzy zdecydowali się na przeprowadzania okresowych testów systemu GGSS, dzięki czemu możliwe było wczesne wykrywanie i eliminowanie pojawiających się w projekcie błędów. Regularne przeprowadzanie weryfikacji poprawności działania najnowszej wersji warstwy oprogramowania systemu GGSS pozwoliło ponadto na wygodne testowanie wprowadzanych przez autorów funkcjonalności - duża częstotliwość oznacza w tym przypadku możliwość testowania niewielkiego zbioru zmian, co znacząco ułatwia wczesne wykrywanie związanych z nimi nieprawidłowości.


% Specyfika (co ile, w jakich przypadkach byly wykonywane)
% Wspomniec ze opisane w inz, testowane rozne wersje, ze dzialaja, ktora uzywana najczesciej
% Co bylo testowane (poprawnosc dzialania przez kilka godzin/kilka dni, pojedyncze funkcjonalnosci w zaleznosci od potrzeb, zuzycie pamieci za pomoca skryptow + listing)
% Opisac krotko procedure (moze jakis diagram)

\section{Testy po migracji systemu (JC)}
% Opisac krotko co to za migracja
% Jak miala wygladac, a jak wygladala
% Wklad w migracje
% Testy hardware + na co sie przydaly
% Uruchomienie glownej aplikacji ggss

\section{Testy wersji finalnej (AK i JC)}
% Wstep, kiedy wykonywane, w jakim czasie (AK)
% Kazda funkcjonalnosc z osobna
%   - po kolei opis kazdej (np. hvkiller, komendy itp)
% Testy hardware (skrypty)
% Testy zuzycia zasobow