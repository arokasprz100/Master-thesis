\clearpage
\section{Using GGSS DIM HV commands}
This file describes new command syntax for CAEN High Voltage Units. All commands are case-insensitive. There are 3 types of supported commands:
\begin{itemize}
    \item \textbf{MON} - for performing get operations, can refer to channels or modules
    \item \textbf{SET} - for setting parameters, can refer to channels or modules
    \item \textbf{RAW} - for backward compatibility with old command syntax
\end{itemize}

\dotfill

\subsubsection*{MON Channel Commands}
\begin{itemize}
    \item \textbf{Syntax:} \lstinline{hv <module_alias>:<channel_number> mon <param[,other_params]>}
    \item \textbf{Output format:} \lstinline{OK: <module_alias>:<channel>:<param>:<value>;[...]}
    \item \textbf{Supported parameters:} VSET, VMIN VMAX, VDEC, VMON, ISET, IMIN, IMAX, ISDEC, IMON,MAXV, MVMIN, MVMAX, MVDEC, RUP, RDW, PDWN, RUPMIN, RUPMAX, RUPDEC, RDWMIN, RDWMAX, RDWDEC, TRIP, TRIPMIN, TRIPMAX, TRIPDEC and POL
    \item \textbf{Example:} \lstinline{hv_caen_n1470_0:2 mon vmon,vset} returns VMON and VSET for given channel (number 2) and module (described by alias \lstinline{hv_caen_n1470_0})
    \item \textbf{Example output:} \lstinline{OK: hv_caen_n1470_0:2:VMON:1374;hv_caen_n1470_0:2:VSET:1374;}
    \item One can use \lstinline{*} character to specify that command should be performed for all modules/channels. Example: \lstinline{hv *:* mon vmon}
    \item One can use \lstinline{*} to get output for following parameters: VSET, VMON, ISET, IMON, RUP, RDW, TRIP and POL
\end{itemize}

\dotfill

\subsubsection*{MON Module Commands}
\begin{itemize}
    \item \textbf{Syntax:} \lstinline{hv <module_alias> mon <param>}
    \item \textbf{Output format:} \lstinline{OK: <module_alias>:<param>:<value>;[...]}
    \item \textbf{Supported parameters:} BDNAME, BDFREL, BDSNUM, BDILK, IDILKM, BDCTR and BDTERM.
    \item \textbf{Example:} \lstinline{hv hv_caen_n1470_0 mon bdctr} returns Control Mode for given module
    \item \textbf{Example output:} \lstinline{OK: hv_caen_n1470_0:BDCTR:REMOTE;}
    \item One can use \lstinline{*} to specify that all high voltage modules should be used
\end{itemize}

\dotfill

\subsubsection*{SET Channel Commands}
\begin{itemize}
    \item \textbf{Syntax:} \lstinline{hv <module_alias>:<channel> set <param> <value>}
    \item \textbf{Output format:} \lstinline{OK: <module_alias>:<channel>:<param>:<result>;[...]}
    \item \textbf{Supported parameters:} VSET, ISET, MAXV and TRIP
    \item \textbf{Example:} \lstinline{hv hv_caen_n1470_0:1 set vset 1} sets voltage on given module (specified by alias \lstinline{hv_caen_n1470_0}) and channel (number 1) to 1
    \item \textbf{Example output:} \lstinline{OK: hv_caen_n1470_0:1:VSET:OK;}
    \item \textbf{Special case:} enabling/disabling a channel (no value, only parameter: \lstinline{ON} or \lstinline{OFF})
    \item \textbf{Example:} \lstinline{hv hv_caen_n1470_0:1 set on}
    \item One can use \lstinline{*} character to specify that command should be performed for all modules/channels
\end{itemize}

\dotfill

\subsubsection*{SET Module Commands}
\begin{itemize}
    \item \textbf{Syntax:} \lstinline{hv <module_alias> set <param> <value>}
    \item \textbf{Output format:} \lstinline{OK: <module_alias>:<param>:<result>;[...]}
    \item \textbf{Supported parameters:} RAMP
    \item One can use \lstinline{*} character to specify that command should be performed for all modules
\end{itemize}

\dotfill

\subsubsection*{RAW Commands}
\begin{itemize}
    \item \textbf{Syntax:} \lstinline{hv <module_alias> raw <command_content>}
    \item \textbf{Output format:} \lstinline{OK: <module_alias>:<output_from_hv>}
    \item \textbf{Example:} \lstinline{hv hv_caen_n1470_0 raw $BD:00,CMD:MON,CH:0,PAR:IMON}
    \item \textbf{Example output:} \lstinline{OK: hv_caen_n1470_0:#BD:00,CMD:OK,VAL:0000.00}
    \item Module number in the command content must match ID of the module with given \lstinline{<module_alias>}
\end{itemize}
