\chapter{Wybrane poradniki / Selected guides} % po angielsku
\label{cha:howtos}

% wstep po ang i po polsku, czemu sa po angielsku

\section{Adding modules to the project using existing CMake templates}

\clearpage
\section{Working with git submodules}
\label{sec:working-with-git-sudmobules}

\subsection*{Overview}
This file contains tips for working with complex git structure created within ggss project.

\subsection*{Content}
This document contains following tips:
\begin{itemize}
    \item How to clone whole repository structure of the ggss project
    \item How to push changes inside component
    \item How to propagate changes to whole ggss project
\end{itemize}

\subsubsection*{How to clone whole repository structure of the ggss project}
\begin{itemize}
    \item \lstinline{git clone ssh://git@gitlab.cern.ch:7999/atlas-trt-dcs-ggss/ggss-all.git} - to clone main repository
    \item \lstinline{cd ggss-all}
    \item \lstinline{git submodule update --init --recursive} - to clone all descendant repositories
    \item \lstinline{git submodule foreach --recursive git checkout master} - this step is needed because after previous state all submodules heads will be in detached state (1)
\end{itemize}

\subsubsection*{Notes}
\begin{itemize}
    \item \lstinline{M ggss-hardware-libs} - if there is message about modifications it means that parent repository has outdates submodule link. This issue can be solved using gitio to align repositories.
    \item We can clone part of the ggss-project by changing ggss-all to other repository, for instance ggss-hardware-libs. In such case only repositories descendant to ggss-hardware-libs will be cloned.
\end{itemize}

\subsubsection*{How to push changes inside component}
There are few things you have to remember when pushing changes to repository:
\begin{itemize}
    \item Make sure that you are using newest version of the repository (newest commit).
    \item Make sure that your working version is not in HEAD Detached state (git status).
    \item Before aligning whole project using gitio make sure that your changes do not affect other components and/or align the affected components manually. (e.g. changes in library interface that is used by other components)
    \item Make sure to merge changes to master branch ONLY after changes were tested.
    \item Use gitio script ONLY to update links to submodules. Gitio is not intended to push any code changes and it will not allow to do so.
\end{itemize}

\subsubsection*{How to propagate changes to whole ggss project}

As propagating changes within multi-level submodule based project requires many commits (you have to propagate changes in every dependent repository by making new commit) gitio script has been created to handle the process.

Gitio is mentioned to analyze submodules and their dependencies, create tree structure basing on gathered information and update all the submodule pointers to the newest available versions on master branches.

For detailed usage please read README in ggss-gitio repository.

\clearpage
\section{Creating proper GGSS Docker image for CI/CD infrastructure}

\subsection*{How to prepare new Docker image for GGSS}

\subsubsection*{Overview}
This file contains steps needed to create new docker image for GGSS project.

\subsubsection*{Steps}
\begin{itemize}
    \item Preapre proper Dockerimage file to create new image. (Dockerfilefrom ggss-aux repository can be used. Please align library/kernel version to your needs and prepare all needed resources for the Dockerfile e.g.: boost)
    \item Create new image using command \lstinline{docker image build .} in directory containing Dockerfile. Remember image hash which will be visible at the end of the output \lstinline{Successfully built e250289733bd}.
    \item Properly tag new image \lstinline{docker image tag e250289733bd} \lstinline{gitlab-registry.cern.ch/atlas-trt-dcs-ggss/ggss-all/centos7:v<new_version>} and \lstinline{gitlab-registry.cern.ch/atlas-trt-dcs-ggss/ggss-all/centos7:latest}
    \item Login to gitlab docker image registry \lstinline{docker login gitlab-registry.cern.ch}
    \item Push image to the registry: \lstinline{docker push gitlab-registry.cern.ch/atlas-trt-dcs-ggss/} \lstinline{ggss-all/centos7:v<new_version>} and \lstinline{gitlab-registry.cern.ch/atlas-trt-dcs-ggss/} \lstinline{ggss-all/centos7:latest}
    \item Remember to also update kernel version on registered GitLab Runners for the ggss project to match the version installed in newly created image.
\end{itemize}

\section{Using new GGSS DIM commands}
\section{Using GGSS DIM HV commands}