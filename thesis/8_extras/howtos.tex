\chapter{Wybrane poradniki / Selected guides} % po angielsku
\label{cha:howtos}

Niniejszy dodatek zawiera przydatne, zdaniem autorów, poradniki opisujące pewne aspekty pracy z systemem GGSS. Część z nich stanowi rozwinięcie lub zaktualizowaną wersję poradników przygotowanych w ramach pracy inżynierskiej. Z uwagi na fakt, iż zaprezentowane treści mogą zostać wykorzystane w charakterze dokumentacji, poradniki załączone zostały w języku angielskim. Znaczą część z nich znaleźć można w repozytorium \emph{ggss-aux} na platformie GitLab.

This appendix contains useful, in the opinion of the authors, guides on certain aspects of working with the GGSS system. Some of them are an extension or an updated version of the guides prepared as part of the engineering thesis. Due to the fact that the presented content can be used as documentation, the guides are included in English. Most of them can be found in the \emph{ggss-aux} repository on the GitLab platform.

\section{Adding modules to the project using existing CMake templates}
This document describes how to add new static library to the GGSS project using the \lstinline{BuildStaticLibrary.cmake} template that can be found in \emph{ggss-util-libs} repository. To use it, perform following actions:
\begin{itemize}
    \item add path to the template to \lstinline{CMAKE_MODULE_PATH} variable
    \item include \lstinline{BuildStaticLibrary.cmake} using \lstinline{include} statement
    \item call \lstinline{ggss_build_static_library} with all mandatory and any optional arguments: \begin{itemize}
        \item \lstinline{TARGET_NAME} - name of project (library) to be created (mandatory argument)
        \item \lstinline{DEPENDENCY_PREFIX} - contain common part of all dependencies paths (mandatory if any dependencies specified, optional otherwise)
        \item \lstinline{DEPENDENCIES} - list of library dependencies (optional argument)
    \end{itemize}
    \item one should also consider defining \lstinline{BUILD_OUTPUT_DIRECTORY} variable if using this template in larger project
\end{itemize}

\noindent
If target with given name already exists, it will not be created again (there will be no errors) - \lstinline{return()} will be called instead and function execution will silently end. \textbf{Please note} that this function builds all library dependencies and links them to it - there is no need to use both \lstinline{BuildStaticLibrary.cmake} and \lstinline{BuildDependencies.cmake} at the same time. \textbf{Please note} that this function does not handle library tests - they need to be handled separately. Example usage:

\begin{lstlisting}[language=CMake]
# Set path to CMake templates.
set(CMAKE_MODULE_PATH ${PATH_TO_CMAKE_TEMPLATES})

# To access ggss_build_static_library function.
include(BuildStaticLibrary)

# Build target static library.
ggss_build_static_library(
    TARGET_NAME "thread"
    DEPENDENCY_PREFIX "${CMAKE_CURRENT_SOURCE_DIR}/.." 
    DEPENDENCIES "log" "handle"
)
\end{lstlisting}

\clearpage
\section{Working with git submodules and releasing GGSS}
\label{sec:working-with-git-sudmobules}
This file contains tips for working with complex git structure created within GGSS project.

\subsection*{Content}
This document contains following tips:
\begin{itemize}
    \item How to clone whole repository structure of the GGSS project.
    \item How to push changes inside component.
    \item How to propagate changes to whole GGSS project.
    \item How to release new GGSS version.
\end{itemize}

\subsubsection*{How to clone whole repository structure of the GGSS project}
\begin{itemize}
    \item \lstinline{git clone ssh://git@gitlab.cern.ch:7999/atlas-trt-dcs-ggss/ggss-all.git} - to clone main repository
    \item \lstinline{cd ggss-all}
    \item \lstinline{git submodule update --init --recursive} - to clone all descendant repositories
    \item \lstinline{git submodule foreach --recursive git checkout master} - this step is needed because after previous state all submodules heads will be in detached state
\end{itemize}

\subsubsection*{Notes}
\begin{itemize}
    \item \lstinline{M ggss-hardware-libs} - if there is message about modifications it means that parent repository has outdates submodule link. This issue can be solved using gitio to align repositories.
    \item We can clone part of the ggss-project by changing ggss-all to other repository, for instance ggss-hardware-libs. In such case only repositories descendant to ggss-hardware-libs will be cloned.
\end{itemize}

\subsubsection*{How to push changes inside component}
There are few things you have to remember when pushing changes to repository:
\begin{itemize}
    \item Make sure that you are using newest version of the repository (newest commit).
    \item Make sure that your working version is not in HEAD Detached state: \lstinline{git status}. You can attach it again with \lstinline{git checkout master/<other-branch>} command.
    \item Stage all changes for commit \lstinline{git add .} or \lstinline{git rm <file path>}
    \item Create new commit \lstinline{git commit -m <commit message>}
    \item Push changes to master or side branch: \lstinline{git push origin master/<other-branch>}
\end{itemize}

\subsubsection*{How to propagate changes to whole GGSS project}
\begin{itemize}
    \item As propagating changes within multi-level submodule based project requires many commits (you have to propagate changes in every dependent repository by making new commit) gitio script has been created to handle the process.
    \item Gitio is meant to analyze submodules and their dependencies, create tree structure basing on gathered information and update all the submodule pointers to the newest available versions on master branches.
    \item For detailed usage please read README in gitio repository.
    \item \textbf{Important} \lstinline{gitio} requires Python3 installed to be working properly. Install \lstinline{python3-pip} package (\lstinline{sudo yum install -y python3-pip}) on your machine to use the script. You may also consider using \lstinline{virtualenv}.
\end{itemize}

\subsubsection*{Gitio quick usage guide:}
\begin{itemize}
    \item Clone whole GGSS project or use already cloned repositories.
    \item Clone \lstinline{ggss-gitio} repository: \\\lstinline{git clone ssh://git@gitlab.cern.ch:7999/atlas-trt-dcs-ggss/ggss-gitio.git}
    \item Enter \lstinline{ggss-gitio} directory \lstinline{cd ggss-gitio}
    \item (Optional) Install requirements if they are not preset in your environment \lstinline{pip3 install -r requirements.txt}
    \item Use \lstinline{ggss-gitio} to align repositories \lstinline{python3 gitio.py -p <path-to-repository-root>} (repository root may be relative path to \lstinline{ggss-all} repository e.g.: \lstinline{../ggss-all})
\end{itemize}


\subsubsection*{Notes}
There are few things you have to remember when pushing changes to repository:
\begin{itemize}
    \item Make sure that you are using newest version of the repository (newest commit).
    \item Make sure that your working version is not in HEAD Detached state (git status).
    \item Use gitio script ONLY to update links to submodules. Gitio is not intended to push any code changes and it will not allow to do so.
\end{itemize}

\subsubsection*{How to release new GGSS version}
\begin{itemize}
    \item To release new GGSS project version all changes that are mentioned to be within new release should be pushed to \lstinline{master} branches in corresponding repositories.
    \item All repositories should be aligned using \lstinline{ggss-gitio} script.
    \item After all changes were aligned and pipeline in \lstinline{ggss-all} repository has finished: \begin{itemize}
        \item Create new empty commit with message that follows eslint-semantic-release (\href{https://github.com/conventional-changelog/conventional-changelog/tree/master/packages/conventional-changelog-eslint}{\textbf{link}}) convention \lstinline{git commit -m <commit-message> --allow-empty}
        \item Push the commit to remote \lstinline{git push origin master}
    \end{itemize}
    \item After pipeline for the created commit finishes release should be available \href{https://gitlab.cern.ch/atlas-trt-dcs-ggss/ggss-all/-/releases}{\textbf{here}} 
\end{itemize}
\clearpage
\section{Creating proper GGSS Docker image for CI/CD infrastructure}
This file contains steps needed to create new docker image for GGSS project.

\subsubsection*{Steps}
\begin{itemize}
    \item Prepare proper Docker image file to create new image. (Dockerfile from ggss-aux repository can be used. Please align library/kernel version to your needs and prepare all needed resources for the Dockerfile e.g.: Boost)
    \item Create new image using command \lstinline{docker image build .} in directory containing Dockerfile. Remember image hash which will be visible at the end of the output \lstinline{Successfully built e250289733bd}.
    \item Properly tag new image:
    \lstinline{docker image tag e250289733bd} \\
    \lstinline{gitlab-registry.cern.ch/atlas-trt-dcs-ggss/ggss-all/centos7:v<new_version>} \\
    and \lstinline{gitlab-registry.cern.ch/atlas-trt-dcs-ggss/ggss-all/centos7:latest}
    \item Login to gitlab docker image registry \lstinline{docker login gitlab-registry.cern.ch}
    \item Push image to the registry: \lstinline{docker push gitlab-registry.cern.ch/atlas-trt-dcs-ggss/} \lstinline{ggss-all/centos7:v<new_version>} and \lstinline{gitlab-registry.cern.ch/atlas-trt-dcs-ggss/} \lstinline{ggss-all/centos7:latest}
    \item Remember to also update kernel version on registered GitLab Runners for the ggss project to match the version installed in newly created image.
\end{itemize}
\clearpage
\section{Using CERN Openstack and adding GitLab runner for GGSS project}
This file describes how to use CERN Openstack to create and setup personal VM that can be used as GitLab Runner in GGSS CI/CD environment and contains steps needed.

\subsubsection*{Create VM using CERN Openstack}
\begin{itemize}
    \item Go to CERN Openstack website (\href{https://openstack.cern.ch/}{\textbf{link}})
    \item Choose \lstinline{Project > Compute > Instances > Launch Instance}
    \item In \lstinline{Details} fill: \begin{itemize}
        \item Instance Name (Hostname)
        \item Description
    \end{itemize}
    \item In \lstinline{Source} choose \lstinline{CC7 - x86_64} image
    \item In \lstinline{Flavor} choose \lstinline{m2.large}
    \item Click \lstinline{LAUNCH INSTANCE} in the bottom-right corner
    \item After few minutes new VM will be created and IP address will be visible in \lstinline{Project > Compute > Instances} panel.
    \item To register the new VM as GitLab Runner follow proper guide.
\end{itemize}

\subsubsection*{GGSS project-specific notes}
\begin{itemize}
    \item \textit{IMPORTANT} Remember to install proper kernel version on your Virtual Machine (it should be the same as in production environment and the same as specified in Dockerfile)
    \item Remember to use \lstinline{ggss-builder} tag while registering GitLab Runner
    \item \textit{In case of CERN Openstack VM}: Remember to change path from afs to local VM: \lstinline{sudo mkdir /gitlab-runner && cd /gitlab-runner}
\end{itemize}

\subsubsection*{Set up Docker}

\noindent
Remove old docker versions:
\begin{lstlisting}
sudo yum remove docker \
docker-client \
docker-client-latest \
docker-common \
docker-latest \
docker-latest-logrotate \
docker-logrotate \
docker-engine
\end{lstlisting}

\clearpage
\subsubsection*{Install new docker engine and CLI using official repository}

\noindent
Install required packages:
\begin{lstlisting}
sudo yum install -y yum-utils \
device-mapper-persistent-data \
lvm2
\end{lstlisting}

\noindent
Set up the repository:
\begin{lstlisting}
sudo yum-config-manager \
--add-repo \
https://download.docker.com/linux/centos/docker-ce.repo
\end{lstlisting}

\noindent
Install Docker Engine and CLI:
\begin{lstlisting}
sudo yum install -y docker-ce docker-ce-cli containerd.io
\end{lstlisting}

\noindent
Start Docker:
\begin{lstlisting}
sudo systemctl start docker
\end{lstlisting}

\noindent
Verify if Docker works properly:
\begin{lstlisting}
sudo docker run hello-world
\end{lstlisting}

\subsubsection*{Register virtual machine as a GitLab CI/CD runner}

\noindent
Download appropriate packages:
\begin{lstlisting}
curl -LJO \
https://gitlab-runner-downloads.s3.amazonaws.com/latest/rpm/gitlab-runner_amd64.rpm
\end{lstlisting}

\noindent
Install the packages:
\begin{lstlisting}
sudo yum install -y gitlab-runner_amd64.rpm
\end{lstlisting}

\noindent
\textit{Note: If you want to update runner packages use following command instead:}
\begin{lstlisting}
rpm -Uvh gitlab-runner_amd64.rpm
\end{lstlisting}

\noindent
Download gitlab-runner binary file:
\begin{lstlisting}
sudo curl -L --output /usr/local/bin/gitlab-runner \
gitlab-runner-downloads.s3.amazonaws.com/latest/binaries/gitlab-runner-linux-amd64
\end{lstlisting}

\noindent
Give proper permissions for the binary file:
\begin{lstlisting}
sudo chmod +x /usr/local/bin/gitlab-runner
\end{lstlisting}

\noindent
Create a GitLab CI user:
\begin{lstlisting}
sudo useradd --comment 'GitLab Runner' --create-home gitlab-runner --shell /bin/bash
\end{lstlisting}

\noindent
Install application and run as a service:
\begin{lstlisting}
sudo gitlab-runner install --user=gitlab-runner\
--working-directory=/home/gitlab-runner
sudo gitlab-runner start
\end{lstlisting}

\subsubsection*{Register the runner}

\noindent
Register using gitlab-runner:
\begin{lstlisting}
sudo gitlab-runner register
\end{lstlisting}

\noindent
Enter proper GitLab instance URL:
\begin{lstlisting}
Please enter the gitlab-ci coordinator URL (e.g. https://gitlab.com )
https://gitlab.cern.ch
\end{lstlisting}

\noindent
Enter token obtained from your GitLab group page > settings > CI/CD > Runners (\href{https://gitlab.cern.ch/groups/atlas-trt-dcs-ggss/-/settings/ci_cd}{\textbf{link}} for GGSS project):
\begin{lstlisting}
Please enter the gitlab-ci token for this runner
<your_token_here>
\end{lstlisting}

\noindent
Enter a description for the runner (this can be changed later):
\begin{lstlisting}
Please enter the gitlab-ci description for this runner
[hostname] my-runner
\end{lstlisting}

\noindent
Enter proper tags (this can be changed later):
\begin{lstlisting}
Please enter the gitlab-ci tags for this runner (comma separated):
ggss-builder
\end{lstlisting}

\noindent
Enter runner executor:
\begin{lstlisting}
Please enter the executor: ssh, docker+machine, docker-ssh+machine,\
kubernetes, docker, parallels, virtualbox, docker-ssh, shell:
docker
\end{lstlisting}

\noindent
Enter default image (if not defined in \lstinline{.gitlab-ci.yml}):
\begin{lstlisting}
Please enter the Docker image (eg. ruby:2.1):
cern/cc7-base:latest
\end{lstlisting}


\clearpage
\subsubsection*{Compact commands}

\noindent
Setup Docker:
\begin{lstlisting}[basicstyle=\ttfamily\scriptsize]
(sudo yum remove docker \
    docker-client \
    docker-client-latest \
    docker-common \
    docker-latest \
    docker-latest-logrotate \
    docker-logrotate \
    docker-engine \
    &&
sudo yum install -y yum-utils \
    device-mapper-persistent-data \
    lvm2 \
    &&
sudo yum-config-manager \
    --add-repo \
    https://download.docker.com/linux/centos/docker-ce.repo \
    &&
sudo yum install -y docker-ce docker-ce-cli containerd.io \
    &&
sudo systemctl start docker \
    &&
sudo docker run hello-world)
\end{lstlisting}

\noindent
Prepare VM as GitLab-Runner:
\begin{lstlisting}[basicstyle=\ttfamily\scriptsize]
(sudo mkdir /gitlab-runner \
&&
cd /gitlab-runner \
&&
sudo curl -LJO \
https://gitlab-runner-downloads.s3.amazonaws.com/latest/rpm/gitlab-runner_amd64.rpm \
&&
sudo yum install -y gitlab-runner_amd64.rpm \
&&
sudo curl -L --output /usr/local/bin/gitlab-runner \
gitlab-runner-downloads.s3.amazonaws.com/latest/binaries/gitlab-runner-linux-amd64 \
&&
sudo chmod +x /usr/local/bin/gitlab-runner \
&&
sudo useradd --comment 'GitLab Runner' --create-home gitlab-runner --shell /bin/bash \
&&
sudo gitlab-runner install --user=gitlab-runner \
--working-directory=/home/gitlab-runner \
&&
sudo gitlab-runner start \
&&
sudo rm -rf /gitlab-runner)
\end{lstlisting}

\noindent
Register the runner: only available as interactive command.
\clearpage
\section{Using new GGSS DIM commands}
This document describes newly introduced or modified GGSS DIM commands.

\begin{itemize}
    \item \lstinline{update} - updates parameters and data for all channels currently included in performing measurements
    \item \lstinline{update channel <channel_no>} - updates parameters and data for a given channel, for example \lstinline{update channel 0:3}
    \item \lstinline{update all_straws} - updates parameters and data for all channels, even if they are not used to perform measurements
    \item \lstinline{update all} - updates parameters and data for all channels included in performing measurements and updates spectrum for current channel
    \item \lstinline{update spectrum} - performs spectrum update for currently measured channel
    \item \lstinline{get mcaRefreshInterval} - gets value of refresh interval used for clearing MCA buffer
    \item \lstinline{set ggss parameter mcaRefreshInterval value <expected_value} - sets value (integer, in seconds) of MCA buffer refresh/clearing interval
    \item \lstinline{reset channelsOrder} - sets channel order to default value, loaded at the start from config file (requires GGSS to be stopped)
    \item \lstinline{get defaultChannelsOrder} - gets default channel orders, loaded from the config file
    \item \lstinline{set ggss parameter smoothingWindowHalf value <expected_value>} - sets value of the half of window width used for performing histogram smoothing (moving average filter) before finding initial peak position
    \item \lstinline{get smoothingWindowHalf} - returns value of the half of window width used for performing histogram smoothing
    \item \lstinline{set ggss parameters kindOfFit value <fit_type>} now supports more types: \lstinline{GaussFitXe}, \lstinline{GaussFitAr}, \lstinline{Gauss2FitXe} and \lstinline{Gauss2FitAr}
\end{itemize}

\clearpage
\section{Using GGSS DIM HV commands}
This file describes new command syntax for CAEN High Voltage Units. All commands are case-insensitive. There are 3 types of supported commands:
\begin{itemize}
    \item \textbf{MON} - for performing get operations, can refer to channels or modules
    \item \textbf{SET} - for setting parameters, can refer to channels or modules
    \item \textbf{RAW} - for backward compatibility with old command syntax
\end{itemize}

\dotfill

\subsubsection*{MON Channel Commands}
\begin{itemize}
    \item \textbf{Syntax:} \lstinline{hv <module_alias>:<channel_number> mon <param[,other_params]>}
    \item \textbf{Output format:} \lstinline{OK: <module_alias>:<channel>:<param>:<value>;[...]}
    \item \textbf{Supported parameters:} VSET, VMIN VMAX, VDEC, VMON, ISET, IMIN, IMAX, ISDEC, IMON,MAXV, MVMIN, MVMAX, MVDEC, RUP, RDW, PDWN, RUPMIN, RUPMAX, RUPDEC, RDWMIN, RDWMAX, RDWDEC, TRIP, TRIPMIN, TRIPMAX, TRIPDEC and POL
    \item \textbf{Example:} \lstinline{hv_caen_n1470_0:2 mon vmon,vset} returns VMON and VSET for given channel (number 2) and module (described by alias \lstinline{hv_caen_n1470_0})
    \item \textbf{Example output:} \lstinline{OK: hv_caen_n1470_0:2:VMON:1374;hv_caen_n1470_0:2:VSET:1374;}
    \item One can use \lstinline{*} character to specify that command should be performed for all modules/channels. Example: \lstinline{hv *:* mon vmon}
    \item One can use \lstinline{*} to get output for following parameters: VSET, VMON, ISET, IMON, RUP, RDW, TRIP and POL
\end{itemize}

\dotfill

\subsubsection*{MON Module Commands}
\begin{itemize}
    \item \textbf{Syntax:} \lstinline{hv <module_alias> mon <param>}
    \item \textbf{Output format:} \lstinline{OK: <module_alias>:<param>:<value>;[...]}
    \item \textbf{Supported parameters:} BDNAME, BDFREL, BDSNUM, BDILK, IDILKM, BDCTR and BDTERM.
    \item \textbf{Example:} \lstinline{hv hv_caen_n1470_0 mon bdctr} returns Control Mode for given module
    \item \textbf{Example output:} \lstinline{OK: hv_caen_n1470_0:BDCTR:REMOTE;}
    \item One can use \lstinline{*} to specify that all high voltage modules should be used
\end{itemize}

\dotfill

\subsubsection*{SET Channel Commands}
\begin{itemize}
    \item \textbf{Syntax:} \lstinline{hv <module_alias>:<channel> set <param> <value>}
    \item \textbf{Output format:} \lstinline{OK: <module_alias>:<channel>:<param>:<result>;[...]}
    \item \textbf{Supported parameters:} VSET, ISET, MAXV and TRIP
    \item \textbf{Example:} \lstinline{hv hv_caen_n1470_0:1 set vset 1} sets voltage on given module (specified by alias \lstinline{hv_caen_n1470_0}) and channel (number 1) to 1
    \item \textbf{Example output:} \lstinline{OK: hv_caen_n1470_0:1:VSET:OK;}
    \item \textbf{Special case:} enabling/disabling a channel (no value, only parameter: \lstinline{ON} or \lstinline{OFF})
    \item \textbf{Example:} \lstinline{hv hv_caen_n1470_0:1 set on}
    \item One can use \lstinline{*} character to specify that command should be performed for all modules/channels
\end{itemize}

\dotfill

\subsubsection*{SET Module Commands}
\begin{itemize}
    \item \textbf{Syntax:} \lstinline{hv <module_alias> set <param> <value>}
    \item \textbf{Output format:} \lstinline{OK: <module_alias>:<param>:<result>;[...]}
    \item \textbf{Supported parameters:} RAMP
    \item One can use \lstinline{*} character to specify that command should be performed for all modules
\end{itemize}

\dotfill

\subsubsection*{RAW Commands}
\begin{itemize}
    \item \textbf{Syntax:} \lstinline{hv <module_alias> raw <command_content>}
    \item \textbf{Output format:} \lstinline{OK: <module_alias>:<output_from_hv>}
    \item \textbf{Example:} \lstinline{hv hv_caen_n1470_0 raw $BD:00,CMD:MON,CH:0,PAR:IMON}
    \item \textbf{Example output:} \lstinline{OK: hv_caen_n1470_0:#BD:00,CMD:OK,VAL:0000.00}
    \item Module number in the command content must match ID of the module with given \lstinline{<module_alias>}
\end{itemize}
