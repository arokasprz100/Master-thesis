\clearpage
\section{Using CERN Openstack and adding GitLab runner for GGSS project}
This file describes how to use CERN Openstack to create and setup personal VM that can be used as GitLab Runner in GGSS CI/CD environment and contains steps needed.

\subsubsection*{Create VM using CERN Openstack}
\begin{itemize}
    \item Go to CERN Openstack website (\href{https://openstack.cern.ch/}{\textbf{link}})
    \item Choose \lstinline{Project > Compute > Instances > Launch Instance}
    \item In \lstinline{Details} fill: \begin{itemize}
        \item Instance Name (Hostname)
        \item Description
    \end{itemize}
    \item In \lstinline{Source} choose \lstinline{CC7 - x86_64} image
    \item In \lstinline{Flavor} choose \lstinline{m2.large}
    \item Click \lstinline{LAUNCH INSTANCE} in the bottom-right corner
    \item After few minutes new VM will be created and IP address will be visible in \lstinline{Project > Compute > Instances} panel.
    \item To register the new VM as GitLab Runner follow proper guide.
\end{itemize}

\subsubsection*{GGSS project-specific notes}
\begin{itemize}
    \item \textit{IMPORTANT} Remember to install proper kernel version on your Virtual Machine (it should be the same as in production environment and the same as specified in Dockerfile)
    \item Remember to use \lstinline{ggss-builder} tag while registering GitLab Runner
    \item \textit{In case of CERN Openstack VM}: Remember to change path from afs to local VM: \lstinline{sudo mkdir /gitlab-runner && cd /gitlab-runner}
\end{itemize}

\subsubsection*{Set up Docker}

\noindent
Remove old docker versions:
\begin{lstlisting}
sudo yum remove docker \
docker-client \
docker-client-latest \
docker-common \
docker-latest \
docker-latest-logrotate \
docker-logrotate \
docker-engine
\end{lstlisting}

\clearpage
\subsubsection*{Install new docker engine and CLI using official repository}

\noindent
Install required packages:
\begin{lstlisting}
sudo yum install -y yum-utils \
device-mapper-persistent-data \
lvm2
\end{lstlisting}

\noindent
Set up the repository:
\begin{lstlisting}
sudo yum-config-manager \
--add-repo \
https://download.docker.com/linux/centos/docker-ce.repo
\end{lstlisting}

\noindent
Install Docker Engine and CLI:
\begin{lstlisting}
sudo yum install -y docker-ce docker-ce-cli containerd.io
\end{lstlisting}

\noindent
Start Docker:
\begin{lstlisting}
sudo systemctl start docker
\end{lstlisting}

\noindent
Verify if Docker works properly:
\begin{lstlisting}
sudo docker run hello-world
\end{lstlisting}

\subsubsection*{Register virtual machine as a GitLab CI/CD runner}

\noindent
Download appropriate packages:
\begin{lstlisting}
curl -LJO \
https://gitlab-runner-downloads.s3.amazonaws.com/latest/rpm/gitlab-runner_amd64.rpm
\end{lstlisting}

\noindent
Install the packages:
\begin{lstlisting}
sudo yum install -y gitlab-runner_amd64.rpm
\end{lstlisting}

\noindent
\textit{Note: If you want to update runner packages use following command instead:}
\begin{lstlisting}
rpm -Uvh gitlab-runner_amd64.rpm
\end{lstlisting}

\noindent
Download gitlab-runner binary file:
\begin{lstlisting}
sudo curl -L --output /usr/local/bin/gitlab-runner \
gitlab-runner-downloads.s3.amazonaws.com/latest/binaries/gitlab-runner-linux-amd64
\end{lstlisting}

\noindent
Give proper permissions for the binary file:
\begin{lstlisting}
sudo chmod +x /usr/local/bin/gitlab-runner
\end{lstlisting}

\noindent
Create a GitLab CI user:
\begin{lstlisting}
sudo useradd --comment 'GitLab Runner' --create-home gitlab-runner --shell /bin/bash
\end{lstlisting}

\noindent
Install application and run as a service:
\begin{lstlisting}
sudo gitlab-runner install --user=gitlab-runner\
--working-directory=/home/gitlab-runner
sudo gitlab-runner start
\end{lstlisting}

\subsubsection*{Register the runner}

\noindent
Register using gitlab-runner:
\begin{lstlisting}
sudo gitlab-runner register
\end{lstlisting}

\noindent
Enter proper GitLab instance URL:
\begin{lstlisting}
Please enter the gitlab-ci coordinator URL (e.g. https://gitlab.com )
https://gitlab.cern.ch
\end{lstlisting}

\noindent
Enter token obtained from your GitLab group page > settings > CI/CD > Runners (\href{https://gitlab.cern.ch/groups/atlas-trt-dcs-ggss/-/settings/ci_cd}{\textbf{link}} for GGSS project):
\begin{lstlisting}
Please enter the gitlab-ci token for this runner
<your_token_here>
\end{lstlisting}

\noindent
Enter a description for the runner (this can be changed later):
\begin{lstlisting}
Please enter the gitlab-ci description for this runner
[hostname] my-runner
\end{lstlisting}

\noindent
Enter proper tags (this can be changed later):
\begin{lstlisting}
Please enter the gitlab-ci tags for this runner (comma separated):
ggss-builder
\end{lstlisting}

\noindent
Enter runner executor:
\begin{lstlisting}
Please enter the executor: ssh, docker+machine, docker-ssh+machine,\
kubernetes, docker, parallels, virtualbox, docker-ssh, shell:
docker
\end{lstlisting}

\noindent
Enter default image (if not defined in \lstinline{.gitlab-ci.yml}):
\begin{lstlisting}
Please enter the Docker image (eg. ruby:2.1):
cern/cc7-base:latest
\end{lstlisting}


\clearpage
\subsubsection*{Compact commands}

\noindent
Setup Docker:
\begin{lstlisting}[basicstyle=\ttfamily\scriptsize]
(sudo yum remove docker \
    docker-client \
    docker-client-latest \
    docker-common \
    docker-latest \
    docker-latest-logrotate \
    docker-logrotate \
    docker-engine \
    &&
sudo yum install -y yum-utils \
    device-mapper-persistent-data \
    lvm2 \
    &&
sudo yum-config-manager \
    --add-repo \
    https://download.docker.com/linux/centos/docker-ce.repo \
    &&
sudo yum install -y docker-ce docker-ce-cli containerd.io \
    &&
sudo systemctl start docker \
    &&
sudo docker run hello-world)
\end{lstlisting}

\noindent
Prepare VM as GitLab-Runner:
\begin{lstlisting}[basicstyle=\ttfamily\scriptsize]
(sudo mkdir /gitlab-runner \
&&
cd /gitlab-runner \
&&
sudo curl -LJO \
https://gitlab-runner-downloads.s3.amazonaws.com/latest/rpm/gitlab-runner_amd64.rpm \
&&
sudo yum install -y gitlab-runner_amd64.rpm \
&&
sudo curl -L --output /usr/local/bin/gitlab-runner \
gitlab-runner-downloads.s3.amazonaws.com/latest/binaries/gitlab-runner-linux-amd64 \
&&
sudo chmod +x /usr/local/bin/gitlab-runner \
&&
sudo useradd --comment 'GitLab Runner' --create-home gitlab-runner --shell /bin/bash \
&&
sudo gitlab-runner install --user=gitlab-runner \
--working-directory=/home/gitlab-runner \
&&
sudo gitlab-runner start \
&&
sudo rm -rf /gitlab-runner)
\end{lstlisting}

\noindent
Register the runner: only available as interactive command.