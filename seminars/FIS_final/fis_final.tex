\documentclass[10pt]{beamer}

\usepackage[english]{babel}
\usepackage[utf8]{inputenc}
\usepackage{lmodern}
\usepackage{listings}
\usepackage{caption}
\usepackage{subcaption}

\captionsetup[lstlisting]{ margin=0pt }

\definecolor{lgray}{gray}{0.96}
\definecolor{lbcolor}{rgb}{0.9,0.9,0.9}
\lstset{
    framesep=2pt,
    basicstyle=\ttfamily\scriptsize,
    breaklines=true,
    breakatwhitespace=true,
    aboveskip={0.75\baselineskip},
    columns=fixed,
    showstringspaces=false,
    breaklines=true,
    frame=single,
    rulecolor=\color{lgray},
    showtabs=false,
    showspaces=false,
    showstringspaces=false,
    backgroundcolor=\color{lgray},
    identifierstyle=\ttfamily,
    keywordstyle=\color[rgb]{0,0,1},
    commentstyle=\color[rgb]{0.0,0.26,0.15},
    stringstyle=\color[rgb]{0.627,0.126,0.941}
}

\setbeamersize{text margin left=5mm,text margin right=5mm}


\usetheme{AGH}
\title{Rozbudowa i uaktualnienie systemu GGSS detektora ATLAS TRT}
\author{\normalsize{Arkadiusz Kasprzak \newline \and 
    Jarosław Cierpich \newline \newline \and 
    Opiekun pracy: dr hab. inż. Bartosz Mindur, prof. AGH}}
\date{}

\begin{document}

\titleframe[pl]

\part{Prezentacja - Arkadiusz Kasprzak}

\begin{frame}
\frametitle{Plan prezentacji - Arkadiusz Kasprzak}
\tableofcontents
\end{frame}


\section{Wprowadzenie do tematyki pracy}

\begin{frame}
\frametitle{Wprowadzenie do systemu GGSS}
% domena (CERN, TRT, DCS)
% architektura wysokopoziomowa
% warstwa sprzetowa i oprogramowania
% glowne technologie
\end{frame}

\begin{frame}
\frametitle{Cele pracy}
% kontyuacja pracy inz
% zwrocic uwage na migracje i wyjazd, powiedziec ze prace zdalnie
\end{frame}

\section{Charakterystyka przeprowadzonych prac}

\begin{frame}
\frametitle{Środowisko docelowe i jego ograniczenia}
% wersje kompilatorow i narzedzi
% sposob wdrazania
% itd
\end{frame}

\begin{frame}
\frametitle{Dodatkowe wymagania}
% niezawodnosc (+ stosowane praktyki, np. code review)
% niski prog wejscia
% latwosc i szybkosc wdrozenia zmian
\end{frame}

\section{Modyfikacja kodu źródłowego projektu}

\begin{frame}
\frametitle{Aplikacja \emph{ggss-runner}}
% skrotowy opis dzialania
% wersja pierwotna - charakterystyka
\end{frame}

\begin{frame}
\frametitle{Modernizacja i poprawa jakości kodu źródłowego}
% migracja do cpp11
% poprawa błedów
\end{frame}

\begin{frame}
\frametitle{Wprowadzone funkcjonalności}
% hvcommands
% fitlib
% komendy
% mca buffor
% kolejnosc licznikow
% pomniejsze zmiany
\end{frame}

\begin{frame}
\frametitle{Przykład: nowe komendy dla zasilaczy}
\end{frame}

\begin{frame}
\frametitle{Testy automatyczne}
\end{frame}


\section{Modyfikacja systemu budowania projektu}

\begin{frame}
\frametitle{Modyfikacja systemu budowania projektu}
% cmake i czemu
% co zostalo zmienione
% wsparcie dla testow i dokumentacji
\end{frame}

\section{Podsumowanie}

\begin{frame}
\frametitle{Podsumowanie}
% ile bibliotek zmienionych
% jak praktyki typu testy automatyczne wplynely na prace
% ze dziala od dluzszego czasu
\end{frame}


\part{Prezentacja - Jarosław Cierpich}

\begin{frame}
\frametitle{Plan prezentacji - Jarosław Cierpich}
\tableofcontents
\end{frame}

\section{Modyfikacja architektury projektu}

\begin{frame}
\frametitle{Modyfikacja architektury projektu}
% zmiany w strukturze submodułów
% usunięcie/archiwizacja niektorzych modułów
\end{frame}


\section{Modyfikacja infrastruktury projektu}

\begin{frame}
\frametitle{Mechanizm ciągłej integracji i dostarczania}
% co to jest
% dodanie kroku test do wczesniejszych (build itd)
% artefakty
\end{frame}

\begin{frame}
\frametitle{Automatyzacja i centralizacja wersjonowania projektu}
% semantic-versioning i semantic-release
% zmiany w skryptach, cmake i ci/cd
\end{frame}

\begin{frame}
\frametitle{Automatyzacja pracy z submodułami}
% zarys problemu
% rozwiazanie problemu
\end{frame}

\begin{frame}
\frametitle{Pozostałe zmiany}
% skrypty operacyjne
% skrypt monitorujacy
% uprzatniecie maszyny docelowej
\end{frame}

\section{Aplikacje do testów urządzeń}

\begin{frame}
\frametitle{Aplikacje do testów urządzeń}
% motywacja + migracja na nowy komputer i hub
% nowe rozwiazanie i jego interfejs (tryb scenariuszowy, brak koniecznosci znajomosci kodu)
% przyklad scenariusza (?)
\end{frame}

\section{Testy projektu}

\begin{frame}
\frametitle{Testy projektu}
% ze byly rozne (cyklicznie i na koniec)
% testy zasobów
% wspomniec o migracji
\end{frame}

\section{Podsumowanie}

\begin{frame}
\frametitle{Podsumowanie}
% realizowane wszystkie wymagania
\end{frame}


\end{document}