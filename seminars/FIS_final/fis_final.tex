\documentclass[10pt]{beamer}

\usepackage[english]{babel}
\usepackage[utf8]{inputenc}
\usepackage{lmodern}
\usepackage{listings}
\usepackage{caption}
\usepackage{subcaption}

\captionsetup[lstlisting]{ margin=0pt }

\definecolor{lgray}{gray}{0.96}
\definecolor{lbcolor}{rgb}{0.9,0.9,0.9}
\lstset{
    framesep=2pt,
    basicstyle=\ttfamily\scriptsize,
    breaklines=true,
    breakatwhitespace=true,
    aboveskip={0.75\baselineskip},
    columns=fixed,
    showstringspaces=false,
    breaklines=true,
    frame=single,
    rulecolor=\color{lgray},
    showtabs=false,
    showspaces=false,
    showstringspaces=false,
    backgroundcolor=\color{lgray},
    identifierstyle=\ttfamily,
    keywordstyle=\color[rgb]{0,0,1},
    commentstyle=\color[rgb]{0.0,0.26,0.15},
    stringstyle=\color[rgb]{0.627,0.126,0.941}
}

\setbeamersize{text margin left=5mm,text margin right=5mm}


\usetheme{AGH}
\title{Rozbudowa i uaktualnienie systemu GGSS detektora ATLAS TRT}
\author{\normalsize{Arkadiusz Kasprzak \newline \and 
    Jarosław Cierpich \newline \newline \and 
    Opiekun pracy: dr hab. inż. Bartosz Mindur, prof. AGH}}
\date{}

\begin{document}

\titleframe[pl]

\part{Prezentacja - Arkadiusz Kasprzak}

\begin{frame}
\frametitle{Plan prezentacji - Arkadiusz Kasprzak}
\tableofcontents
\end{frame}


\section{Wprowadzenie do tematyki pracy}

\begin{frame}
\frametitle{Wprowadzenie do systemu GGSS}
\begin{itemize}
    \item System Stabilizacji Wzmocnienia Gazowego (GGSS - \emph{Gas Gain Stabilization System})
    \item projekt zintegrowany z systemem kontroli detektora ATLAS w CERN
    \item umożliwia poprawne działanie detektora TRT (\emph{Transition Radiation Tracker}) będącego częścią ATLAS
    \item składa się z warstwy oprogramowania (aplikacje oraz infrastruktura) oraz sprzętu (zasilacz wysokiego napięcia, multiplekser analogowy, wielokanałowy analizator amplitudy MCA)
    \item warstwa oprogramowania koordynuje działanie urządzeń i umożliwia sterowanie nimi za pomocą specjalnych komend
    \item aplikacje i biblioteki napisane z wykorzystaniem języka C++, infrastruktura: CMake, Bash oraz Python
\end{itemize}
% domena (CERN, TRT, DCS)
% architektura wysokopoziomowa
% warstwa sprzetowa i oprogramowania
% glowne technologie
\end{frame}

\begin{frame}
\frametitle{Wprowadzenie do systemu GGSS}
\begin{figure}
\includegraphics[width=0.85\textwidth]{static/high_level_architecture.pdf}
\caption{Wysokopoziomowa architektura systemu GGSS}
\end{figure}
\begin{itemize}
    \item WinCC OA - system system typu SCADA pozwalający na sterowanie i kontrolę działania podsystemów detektora
    \item DIM - protokół komunikacyjny dla aplikacji rozproszonych
\end{itemize}
\end{frame}

\begin{frame}
\frametitle{Cele pracy}
\begin{itemize}
    \item kontynuacja pracy inżynierskiej, skupiającej się na infrastrukturze projektu
    \item rozbudowa przygotowanych w ramach wspomnianej pracy rozwiązań
    \item główny nacisk na aplikację \emph{ggss-runner}, stanowiącą trzon systemu
    \item poprawa jakości kodu źródłowego oraz wprowadzenie nowych funkcjonalności
    \item rozbudowa infrastruktury pozwalającej na testowanie projektu
    \item udokumentowanie projektu (dokumentacja kodu źródłowego oraz pliki instruktażowe)
    \item migracja projektu - nowy komputer docelowy (początkowo planowany wyjazd do CERN, ostatecznie zrealizowana zdalnie)
    \item konieczność zachowania wysokiej niezawodności systemu - stosowanie praktyk takich jak \emph{code review} i testy automatyczne
\end{itemize}
% kontyuacja pracy inz
% zwrocic uwage na migracje i wyjazd, powiedziec ze prace zdalnie
% wymagana niezawodnosc - praktyki (code review itd)
\end{frame}

\begin{frame}
\frametitle{Środowisko docelowe i ograniczenia}
\begin{itemize}
    \item projekt ściśle związany z infrastrukturą CERN
    \item aplikacje wchodzące w skład systemu działają na dedykowanym komputerze z dostępem do wymaganych urządzeń
    \item ograniczenia dotyczące wersji kompilatora języka C++ oraz narzędzia CMake
    \item ograniczone uprawnienia w środowisku docelowym
    \item konieczność zachowania kompatybilności wstecznej
\end{itemize}
% wersje kompilatorow i narzedzi
% sposob wdrazania
% itd
\end{frame}

\section{Modyfikacja kodu źródłowego projektu}

\begin{frame}
\frametitle{Aplikacja \emph{ggss-runner}}
\begin{itemize}
    \item wielowątkowa aplikacja napisana z wykorzystaniem C++ i pakietu Boost
    \item zadanie: cykliczne gromadzenie danych w postaci widma poprzez komunikacje z wielokanałowym analizatorem amplitudy oraz wyznaczanie na ich podstawie wartości napięcia, komunikacja z systemem WinCC OA
\end{itemize}
\begin{figure}
\includegraphics[width=0.6\textwidth]{static/winccoa_panel.png}
\caption{Panel dostępny w ramach systemu WinCC OA}
\end{figure}
% skrotowy opis dzialania
% wersja pierwotna - charakterystyka
\end{frame}

\begin{frame}
\frametitle{Aplikacja \emph{ggss-runner}}
\begin{figure}
\includegraphics[width=0.7\textwidth]{static/flow.pdf}
\caption{Uproszczony schemat działania aplikacji}
\end{figure}
% skrotowy opis dzialania
% wersja pierwotna - charakterystyka
\end{frame}

\begin{frame}
\frametitle{Modernizacja i poprawa jakości kodu źródłowego}
\begin{itemize}
    \item pierwotnie kod wykorzystywał standard C++03 oraz elementy C++11
    \item przeprowadzono migrację do standardu C++11 (pętle zakresowe, silne typy wyliczeniowe itd.)
    \item migracja niepełna z uwagi na ograniczenia kompilatora
    \item ujednolicono konwencje stosowane w kodzie (formatowanie, nazewnictwo)
    \item zlikwidowano niewykorzystywane lub wykomentowane fragmenty kodu
    \item usunięto nieliczne błędy (biblioteki \emph{xml-lib} oraz \emph{log-lib})
    \item znaczne zmiany w 12 z 14 bibliotek wchodzących w skład systemu
\end{itemize}
% migracja do cpp11
% poprawa błedów
\end{frame}

\begin{frame}
\frametitle{Wprowadzone funkcjonalności}
\begin{itemize}
    \item dodanie obsługi zaawansowanych komend dla zasilaczy wysokiego napięcia
    \item rozbudowa biblioteki odpowiedzialnej za dopasowanie krzywej do zebranych danych
    \item dodanie możliwości aktualizacji parametrów i zebranych danych na żądanie
    \item dodanie zabezpieczenia przed przepełnieniem bufora urządzenia MCA
    \item dodanie możliwości przywracania domyślnej kolejności liczników słomkowych
\end{itemize}
\end{frame}

\begin{frame}
\frametitle{Przykład: nowe komendy dla zasilaczy}
\end{frame}

% TODO: moze jakis link do repo zeby pokazac 
\begin{frame}
\frametitle{Testy automatyczne}
\begin{itemize}
    \item w celu zapewnienia niezawodności i szybkiego wykrywanie błędów wykorzystano testy automatyczne i metodykę TDD (\emph{Test Driven Development})
    \item testy przygotowane z wykorzystaniem biblioteki Boost.Test
    \item uruchamiane automatycznie po wdrożeniu zmian do repozytoriów projektu
\end{itemize}
\begin{figure}
\includegraphics[width=0.65\textwidth]{static/pipeline.png}
\caption{\emph{Pipeline CI/CD} wykonujący testy automatyczne po wdrożeniu zmian}
\end{figure}
\end{frame}


\section{Modyfikacja systemu budowania projektu}

\begin{frame}[fragile]
\frametitle{Modyfikacja systemu budowania projektu}
\begin{itemize}
    \item system oparty o narzędzie CMake, przygotowany w ramach pracy inżynierskiej
    \item jego zadaniem jest obsługa projektu o hierarchicznej strukturze
    \item wsparcie dla testów automatycznych oraz generowania dokumentacji za pomocą narzędzia Doxygen
    \item system oparty o tzw. szablony - pliki \emph{.cmake} zawierające często wykorzystywane fragmenty kodu
    \item zmiana sposobu implementacji szablonów - wykorzystanie funkcji i makr
\end{itemize}
\begin{lstlisting}[caption={}]
ggss_build_static_library(
    TARGET_NAME "thread"
    DEPENDENCY_PREFIX "${CMAKE_CURRENT_SOURCE_DIR}/.."
    DEPENDENCIES "sigslot" "ggss-util-libs/log"
)
\end{lstlisting}
% cmake i czemu
% co zostalo zmienione
% wsparcie dla testow i dokumentacji
\end{frame}

\section{Podsumowanie}

\begin{frame}
\frametitle{Podsumowanie}
\begin{itemize}
    \item wszystkie podstawione wymagania zostały spełnione
    \item stosowanie nowoczesnych praktyk takich jak testy automatyczne pozytywnie wpłynęło na proces rozwoju oprogramowania
    \item liczne testy w środowisku docelowym potwierdzają poprawność wprowadzonych zmian
    \item możliwy dalszy rozwój projektu - modernizacja kodu, zmiany na poziomie architektonicznym
\end{itemize}
% ile bibliotek zmienionych
% jak praktyki typu testy automatyczne wplynely na prace
% ze dziala od dluzszego czasu
\end{frame}


\part{Prezentacja - Jarosław Cierpich}

\begin{frame}
\frametitle{Plan prezentacji - Jarosław Cierpich}
\tableofcontents
\end{frame}

\section{Modyfikacja architektury projektu}

\begin{frame}
\frametitle{Modyfikacja architektury projektu}
% zmiany w strukturze submodułów
% usunięcie/archiwizacja niektorzych modułów
\end{frame}


\section{Modyfikacja infrastruktury projektu}

\begin{frame}
\frametitle{Mechanizm ciągłej integracji i dostarczania}
% co to jest
% dodanie kroku test do wczesniejszych (build itd)
% artefakty
\end{frame}

\begin{frame}
\frametitle{Automatyzacja i centralizacja wersjonowania projektu}
% semantic-versioning i semantic-release
% zmiany w skryptach, cmake i ci/cd
\end{frame}

\begin{frame}
\frametitle{Automatyzacja pracy z submodułami}
% zarys problemu
% rozwiazanie problemu
\end{frame}

\begin{frame}
\frametitle{Pozostałe zmiany}
% skrypty operacyjne
% skrypt monitorujacy
% uprzatniecie maszyny docelowej
\end{frame}

\section{Aplikacje do testów urządzeń}

\begin{frame}
\frametitle{Aplikacje do testów urządzeń}
% motywacja + migracja na nowy komputer i hub
% nowe rozwiazanie i jego interfejs (tryb scenariuszowy, brak koniecznosci znajomosci kodu)
% przyklad scenariusza (?)
\end{frame}

\section{Testy projektu}

\begin{frame}
\frametitle{Testy projektu}
% ze byly rozne (cyklicznie i na koniec)
% testy zasobów
% wspomniec o migracji
\end{frame}

\section{Podsumowanie}

\begin{frame}
\frametitle{Podsumowanie}
% realizowane wszystkie wymagania
\end{frame}


\end{document}