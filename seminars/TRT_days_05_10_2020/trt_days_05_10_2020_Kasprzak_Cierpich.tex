\documentclass[10pt]{beamer}

\usepackage[english]{babel}
\usepackage[utf8]{inputenc}
\usepackage{lmodern}
\usepackage{listings}
\usepackage{caption}
\usepackage{subcaption}

\captionsetup[lstlisting]{ margin=0pt }

\definecolor{lgray}{gray}{0.96}
\definecolor{lbcolor}{rgb}{0.9,0.9,0.9}
\lstset{
    framesep=2pt,
    basicstyle=\ttfamily\scriptsize,
    breaklines=true,
    breakatwhitespace=true,
    aboveskip={0.75\baselineskip},
    columns=fixed,
    showstringspaces=false,
    breaklines=true,
    frame=single,
    rulecolor=\color{lgray},
    showtabs=false,
    showspaces=false,
    showstringspaces=false,
    backgroundcolor=\color{lgray},
    identifierstyle=\ttfamily,
    keywordstyle=\color[rgb]{0,0,1},
    commentstyle=\color[rgb]{0.0,0.26,0.15},
    stringstyle=\color[rgb]{0.627,0.126,0.941}
}

\setbeamersize{text margin left=5mm,text margin right=5mm}


\usetheme{AGH}
\title{Core GGSS software update and upgrade}
\subtitle{\normalsize{Tasks undertaken as part of the master's thesis}}
\author{\normalsize{Arkadiusz Kasprzak \newline \and 
    Jarosław Cierpich \newline \newline \and 
    Supervisor: Bartosz Mindur}}
\date{}

\begin{document}

\titleframe[en]

\begin{frame}
\frametitle{Agenda}
\tableofcontents
\end{frame}

\section {Overview of changes}


\begin{frame}
\frametitle{Overview of changes}
\begin{itemize}
    \item C++ codebase refactoring: \begin{itemize}
        \item migration to C++11/14 (range-for loops, uniform initialization etc)
        \item removing old, unused code 
        \item adding more comprehensive documentation
        \item introducing TDD (Test Driven Development)
    \end{itemize}
    \item CMake files refactoring
    \item creating tools for versioning and Git submodule handling
\end{itemize}
\end{frame}


\begin{frame}[fragile]
\frametitle{Migration to C++11/14}
Example of migration to C++11/14 - replacing iterator loop with range-for one. Below You can see the old code.
\begin{lstlisting}[language=c++, caption={Example of old C++ code (before refactoring).}]
for ( XMLTag::NestedTags::const_iterator j = tag.getNestedTags().begin()
            ; j != tag.getNestedTags().end()
            ; j++
        )
{
    if((j->second->getName() == tagName)
        &&(j->second->getAttributeValue("id") == idValue))
        return j->second;
    else
        m_findTagById(*(j->second), tagName, idValue);		
} // endfor
\end{lstlisting}
\end{frame}


\begin{frame}[fragile]
\frametitle{Migration to C++11/14}
\begin{lstlisting}[language=c++, caption={Example of new C++ code (after refactoring).}]
const XMLTag::NestedTags& nestedTags = startingTag.getNestedTags();
for(const auto& nestedTag: nestedTags)
{
    if((nestedTag.second->getName() == tagName) && (nestedTag.second->getAttributeValue("id") == idValue))
    {
        return nestedTag.second;
    }
}
\end{lstlisting}
\begin{itemize}
    \item using range-for loop increases readability of the code
    \item \lstinline[basicstyle=\ttfamily\normalsize]{else} clause has been removed - result of the recursive function call was never used
    \item no need to use the \lstinline[basicstyle=\ttfamily\normalsize]{*} operator
    \item \lstinline[basicstyle=\ttfamily\normalsize]{nestedTag} is a better name than \lstinline[basicstyle=\ttfamily\normalsize]{j}
\end{itemize}
\end{frame}


\begin{frame}[fragile]
\frametitle{Removing old, unused code}
\begin{itemize}
\item The project contained a lot of code (functions/methods) that were never used.
\item Some of them could even be harmful if used.
\item Below example shows two methods that have been removed (why?) from \lstinline[basicstyle=\ttfamily\normalsize]{QueueLimited} class (a queue with size limit).
\end{itemize}
\begin{lstlisting}[language=c++, caption={Example of removed code.}]
// return the whole queue
const std::deque<T>& getQueue () const {
    return c;
}

// return the whole queue
std::deque<T>& getQueue () {
    return c;
}
\end{lstlisting}
\end{frame}


\begin{frame}[fragile]
\frametitle{Introducing Test Driven Development}
\begin{itemize}
\item For unit tests, we are using \lstinline[basicstyle=\ttfamily\normalsize]{Boost.Test}
\item Component are tested during refactoring, we make sure that our changes do not introduce any new bugs.
\item Each component can be tested separately.
\end{itemize}
\begin{lstlisting}[language=c++, caption={Unit test example}]
/**
 * \brief Checks if proper exception is thrown when performing pop() 
 *        operation on empty container.
 */
BOOST_AUTO_TEST_CASE(
    testIfExceptionIsThrownWhenTryingToPopFromEmptyContainer)
{
    QueueLimited<int> queue{};
    BOOST_CHECK_THROW(
        queue.pop(), 
        QueueLimited<int>::ReadEmptyQueueException);
}   
\end{lstlisting}
\end{frame}

\begin{frame}[fragile]
\frametitle{Continous Integration}
\begin{itemize}
\item Unit tests have been integrated into out CI/CD infrastructure.
\end{itemize}
\begin{figure}
\centering
\includegraphics[width=0.5\textwidth]{resources/pipeline_example.png}
\caption{Example of CI pipeline used in the project.}
\end{figure}
\end{frame}


\begin{frame}[fragile]
\frametitle{CMake files refactoring}
\begin{itemize}
\item GGSS uses CMake for managing the build process of the software.
\item CMake is platform and compiler independent.
\item CMake files have been slightly refactored to improve readability by using macros and functions.
\end{itemize}
\begin{lstlisting}[language=c++, caption={Old version of CMake used for building \emph{thread-lib}}]
set(target_name "thread")
if(NOT TARGET ${target_name})
    set(CMAKE_MODULE_PATH "${GGSS_MISC_PATH}")
    include(BuildLibrary)
    include(FindLibraryBoost)
    include(SetupDoxygen)
    include(SetupTests)

    # notice the need to set some variables before including the file
    set(dependency_prefix "${CMAKE_CURRENT_SOURCE_DIR}/..")
    set(dependencies "handle" "log")
    include(BuildDependencies)
endif() 
\end{lstlisting}
\end{frame}

\begin{frame}[fragile]
\frametitle{CMake files refactoring}
\begin{itemize}
\item Instead of including the CMake template files (which just pastes the code), we invoke \lstinline[basicstyle=\ttfamily\normalsize]{ggss_build_library} with named parameters.
\item Unit tests and Doxygen support has been moved to \lstinline[basicstyle=\ttfamily\normalsize]{ggss_build_library} macro, because every library in the project uses them.
\end{itemize}
\begin{lstlisting}[language=c++, caption={New version of CMake used for building \emph{thread-lib}}]
set(CMAKE_MODULE_PATH "${GGSS_MISC_PATH}")
include(BuildLibrary)

ggss_build_library(
    TARGET_NAME "thread"
    DEPENDENCY_PREFIX "${CMAKE_CURRENT_SOURCE_DIR}/.."
    DEPENDENCIES "log" "sigslot"
)
\end{lstlisting}
\end{frame}


\begin{frame}[fragile]
\frametitle{Complex submodule structure handling - scripts}
\begin{itemize}
\item GGSS project tree contains a complex repository structure with many connections between components.
\item To make it easy to properly initialize project structure git submodules are being used.
\end{itemize}
\begin{lstlisting}[language=c++, caption={Initialize project structure with one command.}]
root@host:/# git clone ssh://git@gitlab.cern.ch:7999/atlas-trt-dcs-ggss/ggss-all.git && cd ggss-all && git submodule update --init --recursive
Cloning into '/CERN/ggss-all/ggss-dim-cs'...
Cloning into '/CERN/ggss-all/ggss-driver'...
Cloning into '/CERN/ggss-all/ggss-oper'...
Cloning into '/CERN/ggss-all/ggss-runner'...
Cloning into '/CERN/ggss-all/ggss-spector'...
Cloning into '/CERN/ggss-all/mca-n957'...
Cloning into '/CERN/ggss-all/ggss-dim-cs/external-dim-lib'...
Cloning into '/CERN/ggss-all/ggss-dim-cs/ggss-misc'...
Cloning into '/CERN/ggss-all/ggss-driver/external-n957-lib'...
Cloning into '/CERN/ggss-all/ggss-driver/ggss-misc'...
...(13 lines truncated)
\end{lstlisting}
\end{frame}

\begin{frame}[fragile]
\frametitle{Complex submodule structure handling - scripts}
\begin{itemize}
\item Using submodules requires to take care of commit hashes that are being linked as a submodule.
\item There may be a situation that "parent" repository is not using the latest version of "child" repository.
\end{itemize}
\begin{figure}
    \centering
    \includegraphics[width=0.8\textwidth]{resources/submodules_problem.pdf}
    \caption{Version of submodule differs from version used by parent.}
\end{figure}
\end{frame}

\begin{frame}[fragile]
\frametitle{Complex submodule structure handling - scripts}
\begin{itemize}
\item gitio script is responsible for updating all outdated links between parent and child repositories.
\item The goal is achieved by creating dependency tree of all available repositories.
\item Starting from the bottom of the tree submodules are being aligned (git commands: add, commit, push).
\end{itemize}
\begin{lstlisting}[language=c++, caption={Gitio in action.}]
root@host:/# python gitio.py -p ./ggss-all/
...(17 lines truncated)
INFO - Aligning ./ggss-all/mca-n957 repository
INFO - Aligning ./ggss-all/ggss-dim-cs repository
INFO - Aligning ./ggss-all/ggss-runner repository
INFO - Aligning ./ggss-all/ggss-spector repository
INFO - Aligning ./ggss-all/ggss-oper repository
INFO - Aligning ./ggss-all/ggss-driver repository
INFO - Aligning ./ggss-all repository
INFO - Aligning finished.
\end{lstlisting}
\end{frame}


\begin{frame}[fragile]
\frametitle{Automated versioning}
\begin{itemize}
\item Automated versioning system has been prepared to keep consistent rpm and release versions throughout whole project.
\item Every commit to main repository (ggss-all) is being analyzed. If commit message contains one of specified phrases, new release is being created.
\end{itemize}
\begin{figure}
    \centering
    \includegraphics[width=0.8\textwidth]{resources/commit.PNG}
    \caption{New commit following eslint convention.}
\end{figure}
\begin{figure}
    \centering
    \includegraphics[width=0.8\textwidth]{resources/commit_message_analyze.PNG}
    \caption{Commit message analysis.}
\end{figure}

\end{frame}


\begin{frame}[fragile]
\frametitle{Automated versioning}
\begin{figure}
    \centering
    \includegraphics[width=0.45\textwidth]{resources/new_release.PNG}
    \caption{Newly created release.}
\end{figure}
\end{frame}


\section{Work in progress}


\begin{frame}[fragile]
\frametitle{Work in progress}
\end{frame}

\begin{frame}[c]
\hfill
\begin{center}
\large{
	Thanks for Your attention.
}
\hfill
\end{center}
\end{frame}

\end{document}